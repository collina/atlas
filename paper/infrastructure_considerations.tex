\section{Factors in Assessing Measurement Validity}

\subsection{Precautions and Considerations}

Iran imposes a multi-layer content filtering apparatus through both direct control of network infrastructure, as well as regulatory mandates under telecommunications licensing regimes. At nearly every level of connectivity, traffic transiting across the boundaries of Iran's service providers and exchange points is subject to the possibility of monitoring or disruption. In most cases, primary form of disruption appears to be either inspection of HTTP headers or DPI-based DNS interception and poisoning  in the path of the international gateway administered by the Telecommunications Company of Iran (TCI). While these mechanisms are effective in enforcing the country's broad range of content restrictions, the apparatus has been demonstrated to be blind to requests that deviate slightly from expectations, including TCP-based DNS queries or web requests with improper spacing \cite{aryan2013internet}.

Despite their distribution across a diversity of countries and networks, RIPE Atlas may not fully reflect the Internet as it is experienced by the public, as probes neither fully emulate either the position nor the configuration of an average user. For example, while Iran has blocked access to YouTube continually since it was used to document post-election protests in June 2009, out of 22 probes queries, 10 probes covering 7 ASNs could successfully negotiate a SSL connection to the site. Moreover, Iran has a history of aggressively disrupting access to the Tor network through deep packet inspection, and has long filtered the project's site. However, while social media platforms and circumvention tools, like Balatarin and Ultrasurf, are subject to DNS interception, torproject.org from the perspective of Atlas is not. Based on TCP traceroutes to addresses with torproject.org DNS A record, restrictions on SSL connectivity are instead accomplished through IP blocking implemented within the TCI.

Incongruencies across probes and deviations from observations are an inevitable a product of the high rate of placement of Atlas probes on commercial and academic networks, as well as their use of a non-Windows operating system. These institutions may have alternative network connectivity that is faster and less highly regulated than consumer networks. Additionally, filtering mechanisms may rely catching plaintext requests or headers sent in the clear, such as TLS's SNI string, as their means to restrict access to SSL content. 

\subsection{Consensus Building}

The international distribution of web services, for purposes of performance, failover and load balancing, has created additional complexity in the determining whether answers received over the Internet are genuine. While SSL and DNSSEC utilize third-party trust to validate answers, Certificate Authorities have been previously compromised by state and non-state actors, and DNSSEC is not widely implemented. In order to validate answers within the Atlas network, we use a cross-country comparision of responses to queries. This methodology assumes that states that interfere with connectivity do not coordinate strategies internationally. States and Internet service providers institute different approaches to content-filtering for purposes of localization, infrastructure or even the monetization of blocked traffic. Furthermore, surveillance is more effective when the public is unaware of the practices and technologies employed against them, placing an strong incentive on secrecy and narrow targeting. Therefore, as a simple test of validity, we count the number of countries or ASNs that an answer, such as an A record or a certificate hash, is seen, and find the mean across all answers. Any response with fewer than the mean number of jurisdictions is treated a potentially aberrant and flagged for further investigation.
