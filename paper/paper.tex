% TEMPLATE for Usenix papers, specifically to meet requirements of
%  USENIX '05
% originally a template for producing IEEE-format articles using LaTeX.
%   written by Matthew Ward, CS Department, Worcester Polytechnic Institute.
% adapted by David Beazley for his excellent SWIG paper in Proceedings,
%   Tcl 96
% turned into a smartass generic template by De Clarke, with thanks to
%   both the above pioneers
% use at your own risk.  Complaints to /dev/null.
% make it two column with no page numbering, default is 10 point

% Munged by Fred Douglis <douglis@research.att.com> 10/97 to separate
% the .sty file from the LaTeX source template, so that people can
% more easily include the .sty file into an existing document.  Also
% changed to more closely follow the style guidelines as represented
% by the Word sample file. 

% Note that since 2010, USENIX does not require endnotes. If you want
% foot of page notes, don't include the endnotes package in the 
% usepackage command, below.

% This version uses the latex2e styles, not the very ancient 2.09 stuff.
\documentclass[letterpaper,twocolumn,10pt]{article}

\usepackage{usenix,epsfig,endnotes}
\usepackage{hyperref}
\usepackage{color}
\usepackage[backend=bibtex]{biblatex}
\bibliography{bibliography}

% Don't use monospace URLs.
\urlstyle{sf}

\definecolor{dred}{rgb}{0.8,0,0}
\hypersetup{colorlinks=true,urlcolor=blue,linkcolor=blue,citecolor=blue}

\begin{document}

%don't want date printed
\date{}

%make title bold and 14 pt font (Latex default is non-bold, 16 pt)
\title{\Large \bf Our Paper}

\author{
% {\rm First Name}\\
% First Institution
% \and
% {\rm Second Name}\\
% Second Institution
% \and
% {\rm Third Name}\\
% Third Institution
}

\maketitle

% Use the following at camera-ready time to suppress page numbers.
% Comment it out when you first submit the paper for review.
\thispagestyle{empty}

\subsection*{Abstract}
It is becoming easier to control Internet traffic and routing for governments and ISPs. On the other
hand it is still hard for researchers to find a distributed research volunteered clients to
investigate these filtering. In this paper, we discuss our efforts to find a proper/desired platform
to investigate CASESTUDY and CASESTUDY. We used RIPE ATLAS which is  (talk about benefits) but
(negatives). We propose an algorithm to balance timeliness, diversity , and cost to facilitate the
usage of ripe.

\section{Introduction}
%train of thoughts: talking about the recent popularity of censorship in the world and countires in
%africa and middle easts(cite news article) Then talk about the exsisting methods and briefly critic each onem then
%talk about what were the charactestics we were interested to have in current tools, then talk about
%Ripe atlas as a suitable tool, then talk about their issue related to cost function then talk about
%our contributions.

% talking about the scope of the problem
Tools for discovering censorship are essential considering their importance for journalists and
activists in many countires. Currently, the existaning infruastructures lack representetive(measurment machines) in
regions that they matter the most or they are not easy to use. 

Talking about the importance of RIPE Atlas in the past months. Refer to papers that used RIPE to collect data(\cite{lutu2014understanding} and \cite{brownlee2014searching}).

% Point out what problems the world is facing and how RIPE Atlas can help.

% Quick description of Atlas.
RIPE Atlas~\cite{atlas} is an Internet measurement network which was started in 2010 by RIPE NCC.
All measurement probes which constitute the network are run by volunteers.  Once a user connected
her probe to the network, it can be used for measurements and the user is also able to run custom
measurements by spending some of the network's currency, Atlas credits.  These credits are awarded
automatically based on the uptime of contributed probes.

% What kind of measurements does Atlas allow?
As of today, Atlas allows four types of measurements; ping, traceroute, DNS resolution, and X.509
certificate fetching.  All four measurement types can further be parameterised for more fine-grained
control.

% What are our contributions?  Is there more?
This work has three main contributions.
\begin{itemize}
	\item We evaluate the aptitude of RIPE Atlas for censorship analysis.
	\item We propose an algorithm to balance
	timeliness, diversity, and cost to facilitate effective censorship analysis.\footnote{All our
	code will be publicly available but we redacted the URL for anonymisation.}
	\item We present censorship analysis results based on several months of measurements.
\end{itemize}


\section{Related Work}
\label{related_work}
\begin{table*}[ht!]
\centering
\begin{tabular}{l|cccc}
\textbf{Platform} & \textbf{Flexibility} & \textbf{Coverage} &
\textbf{Blocking resistance} & \textbf{Main use} \\
\hline 
PlanetLab~\cite{planetlab} & High & Low/Medium & Medium & Network measurements \\
Atlas~\cite{atlas} & Low & Medium/High & Medium & Network measurements \\
M-Lab~\cite{dovrolis2010measurement} & Low & High & Medium & Network measurements \\
Tor~\cite{Dingledine2004} & Medium & Medium & Low & Low-latency anonymity \\
OONI~\cite{Filasto2012} & High & Low & Medium & Censorship analysis \\
Herdict~\cite{Herdict} & Low & Low/Medium & Low & Censorship analysis \\
OpenNet~\cite{opennet} & Low & Medium & High & Censorship analysis \\
\hline 
\end{tabular} 
\caption{Comparison between several popular censorship analysis platforms.}
\label{tab:comparison}
\end{table*}

% Longitudinal studies.
It is not difficult to conduct one-off censorship studies because censors
typically do not have sufficient time to react and thwart the research.
Longitudinal studies, on the other hand, are more challenging as they have to
be designed in a tamper-proof and sustainable way.  In 2007, Crandall et al.
proposed ConceptDoppler~\cite{Crandall2007}.  The design enables longitudinal
censorship analysis by detecting which keywords are filtered by the Great
Firewall of China (GFW) over time.  More recently, CensMon was introduced by
Sfakianakis et al. in 2011~\cite{Sfakianakis2011}.  CensMon is a web censorship
monitor which is run on top of PlanetLab~\cite{planetlab}.  Most recently, in
2012, Filast\`{o} and Appelbaum presented OONI~\cite{Filasto2012}.  In
contrast to CensMon and ConceptDoppler, OONI is deployed and has been used
successfully.\footnote{Gathered reports are available online at:
\url{https://ooni.torproject.org/reports/}.}  In parallel to the open platforms
discussed above, proprietary platforms exist~\cite{hwang2007herdict,opennet}.


% Comparison to other projects.
Table~\ref{tab:comparison} contains a comparison between popular and deployed
platforms which are or can be used for censorship analysis.  Our comparison is
based on \emph{flexibility} (i.e., how many types of measurements can be run),
\emph{coverage} (i.e., how many probes in how many countries are available),
and \emph{blocking resistance} (i.e., how easy it is for censors to disable the
respective platform).  We qualitatively compare all platforms and assign them
the labels ``Low'', ``Medium'', or ``High''.  Note that we do not propose
Atlas as \emph{replacement} for any existing censorship measurement platforms.
Instead, we see it as a \emph{complement} that contributes to the already
existing and growing landscape of initiatives.

% Side channel measurements.
% TODO - Are there more papers we are missing?
Additionally, in the absence of deployed platforms or other means to access machines inside
censoring countries, censorship analysts have resorted to exploiting TCP/IP
side channels.  In particular, Ensafi et al. demonstrated that it is possible
to measure intentional packet dropping without controlling either the source or
the destination machine~\cite{Ensafi2014}.

% Previous Atlas censorship analysis.
Atlas has already been used as platform for censorship analysis outside an
academic setting.  In 2014, Maass used Atlas to find inconsistencies in the DNS
records and X.509 certificates for torproject.org~\cite{Maass2014}.  In the
same year, Bortzmeyer and Aben independently discussed Internet censorship in
Turkey~\cite{Bortzmeyer14,Aben14}.  While we discuss the same topic in
Section~\ref{sec:case_studies}, we do so with significantly more data and in a
more rigorous fashion based on the privilege of time.


\section{Design Experiments considering their costs}

The RIPE Atlas credit system works based on a linear cost model. Each user has a know credit balance that can be earned by hosting RIPE Atlas probes, or also by receiving transferred credits. Currently, limited possible type of measurements can be done with known unit costs as described in Table~\ref{tab:cost}. The process of scheduling an experiment starts with each user submit details about the experiment, then the system calculate approximate unit costs. After results are given, the proper amount is subtracted from users credits. Therefore making decisions about how many and what type of measurements can be done is challenging. We develop a command-based application programming interface(API) to help users with their experiment design. 

\subsection{calculating costs}

Predicted cost = $\sum_{i=1}^{5} (C_i * N_i)$  where i can be (DNS, SSL, ...)\\
Remaining credits = min(available cost, desired cost) - predicted cost

Note that it is a linear cost model so number of probs included doesn't matter in the cost part it matters when we suggest the experiment by just uniformly distribute the possible number of measurements on one(total experiments possible/Num probs)
This should be included in the API philipp wrote (TODO)

\subsection{Evaluation of our cost model}
To test that our cost prediction is accurate, we ran a series of unit cost with different values, and compared  the suggested cost and the actual credit subtracted with the cost our approach suggest. Because the number of possible cases are finite we can run all unit measurements, then calculate confidence interval to show how good we do.
\begin{table}[h] 

\hspace{2in}
\centering 
\begin{tabular}{c rr} 
\hline\hline 
Unit Measurement& Type & Cost \\ [0.5ex] 
\hline 
DNS\slash DNS6 & TCP & 20\\ 
DNS\slash DNS6 & UDP & 10\\ 
SSLCert\slash SSLCert6 & & 10 \\
Ping\slash Ping6 & & $N * (int(S/1500)+1)$\\
Traceroute\slash Traceroute6 & & $ 10*N*(int(S/1500)+1)$\\[1ex] 
\hline 
\end{tabular} 
\caption{Measurement Cost Model where N and S are number of packet in the train and packet size, respectively.} 
\label{tab:cost} 
\end{table} 


\cite{Bortzmeyer14,Aben14}
\section{Discussion}

\subsection{Ethical Considerations}

% TODO Roya will fill this up later.
The problem: we might endanger probe contributors by running censored queries over their probes.
Contributors probably don't expect that their probe is used for censorship analysis.

Our justification: Plausible deniability, we are limited to X.509/DNS/ping/traceroute.  We are not
claiming that everything is OK.  Instead, we encourage further discussion.

\subsection{Conclusion}

% The implementation of the March 22nd address blockage of Google Public DNS falsely appears based on timing metrics to be a route hijack. While retaining valid routes across the international frontier into Google's network, the round trip times for several probes dropped precipitiously preceeding their blockage. For one of the last probes blocked, the time taken to reach the last hop out of the country had declined from 60.893 ms to 23.164 ms. However, by March 28,

These findings contribute to broader discussions on anticensorship strategies.  Collateral damage and level of difficulty appears to have shaped the implementation of Turkey and Russia's filtering mandates and responses to circumvention. The quick removal of restrictions on Google Public DNS, and then later attempts to impersonate the service, demonstrated that enforcing an absolute prohibition on filtered content was not worth incurring the cost of disrupting access a large portion of the population. While alternative strategies were possible with Twitter, due to its addressing schema, historical lessons from other countries' attempt to filter YouTube has run into the complexity and interdependency of Google's services. Where there are high collateral costs, such as with YouTube in Turkey and LiveJournal in Russia, authorities have been forced to either limit their restrictions or find cooperative arrangements with the platform. 

Filtering apparatuses may be more effective at disrupting easy access to marginal content or coercing content providers into compliance, than actual denial of information for a sufficiently motivated user. Neither Turkey or Russia's filtering apparatuses appears to have been designed to handle widespread intent to circumvent, particularly when more aggressive restrictions could incur collateral damage. When foreign DNS requests are allowed and not subject to deep packet inspection, circumvention is simple. Upon initial DNS restrictions, only 20\% of probes in Turkey were no longer able to connect to Twitter. The remainder either utilized foreign DNS servers or tunneled traffic out of country by other means, thereby creating inconsistent reports of accessibility. This may reflect a common experience for a large portion of the Turkish population, given previous accounts of the adoption of foreign DNS servers. Direct access to  was only effectively cut off by March 23, when traffic to the platform blocked by IP address. Even with the BGP hijack and local DNS interference, YouTube remained accessible for 40\% of Atlas probes attributed to Turkish networks. 

Finally, due to Turkish and Russian telecommunications companies reliance on blocking network reachability and interfereing with name translation, rather than traffic inspection, the Atlas network was well positioned for both blocking incident. If authorities had utilized HTTP inspection, Atlas would not have been capable of documenting either event. However, without secondary investigation and additional restrictions, initially produced results would have called into question the veracity of accounts of Twitter's filtering. Other inconsistences remain unaccounted for. 

In Atlas's view, several networks within Turk Telekomunikasyon had already began to filter YouTube by the time measurements had been queued at March 21. These measurements precede public accounts and broader censorships by the several days. The differences of experiences of filtering rules is in part a product of Turkey's decentralized network infrastructure, differing Iran and Syria, which maintain monopolistic over international gateways. Content restrictions appear to be instituted by court or administrative order, which are complied with at differing rates. However, this also reflects a disparity between the network conditions of the probes and those of the average user. At the time of blocking, 70\% of probes relied on a DNS server outside of the country, and at least one had at various times tunneled all traffic by unknown means. Moreover, while less than half of measurements within Russia show signs of interference, censorship of the content and manner tested appears to be a persistant experience. Atlas does not reflect the experience of the public portionally based on numerical results, the 20\% of probes experiencing traffic misdirection by Rostelecom is likely representative of a substantially larger percentage of users.

Despite these analytical precautions, RIPE Atlas based measurement provides an early perspective in the opportunities and methodologies possible with pervasive censorship research. Previous examinations of the methods and themes of Internet filtering have tended to analyze specific apparatuses on a per-country basis, assuming internal consistency. This approach has been appropriate for describing the diversity of methods used to control access globally, as well as for when the primary focus is on content themes and in countries that impose restrictions at central points of transit. However, as Internet censorship has increased in locations with heterogentity and private markets at the international frontier, neither compliance strategies nor accountability can assume direct and homogenous control by authorities. Russia and Turkey's networks are more administratively and technically decentralized than China and Iran. The delays or irregulatories in the adoption of practices of particular strategies sheds light on the inconsistences of the application of administrative orders that are applicable to censorship measurement. Our initial research demonstrates that across national networks there are substantive differences of methods, rates of implementation, and, in at least one case, even selective compliance of mandates for content restrictions that are measurable by future initiatives.


\section*{Acknowledgments}

We would like to thank the anonymous reviewers for their constructive
feedback on earlier versions of this paper. Additionally, we would like
to thank Vesna Manojlovic for providing startup credit and the RIPE
Atlas community for their insight. Collin Anderson was supported by the
Internet Policy Observatory program at the University of Pennsylvania's
Annenberg School for Communication.

\raggedright
\printbibliography

\theendnotes

\end{document}
