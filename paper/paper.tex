% TEMPLATE for Usenix papers, specifically to meet requirements of
%  USENIX '05
% originally a template for producing IEEE-format articles using LaTeX.
%   written by Matthew Ward, CS Department, Worcester Polytechnic Institute.
% adapted by David Beazley for his excellent SWIG paper in Proceedings,
%   Tcl 96
% turned into a smartass generic template by De Clarke, with thanks to
%   both the above pioneers
% use at your own risk.  Complaints to /dev/null.
% make it two column with no page numbering, default is 10 point

% Munged by Fred Douglis <douglis@research.att.com> 10/97 to separate
% the .sty file from the LaTeX source template, so that people can
% more easily include the .sty file into an existing document.  Also
% changed to more closely follow the style guidelines as represented
% by the Word sample file. 

% Note that since 2010, USENIX does not require endnotes. If you want
% foot of page notes, don't include the endnotes package in the 
% usepackage command, below.

% This version uses the latex2e styles, not the very ancient 2.09 stuff.
\documentclass[letterpaper,twocolumn,10pt]{article}

\usepackage{usenix,epsfig,endnotes}
\usepackage{hyperref}
\usepackage{color}
\usepackage[backend=bibtex]{biblatex}
\bibliography{bibliography}

% Don't use monospace URLs.
\urlstyle{sf}

\definecolor{dred}{rgb}{0.8,0,0}
\hypersetup{colorlinks=true,urlcolor=blue,linkcolor=blue,citecolor=blue}

\begin{document}

%don't want date printed
\date{}

%make title bold and 14 pt font (Latex default is non-bold, 16 pt)
\title{\Large \bf Our Paper}

\author{
% {\rm First Name}\\
% First Institution
% \and
% {\rm Second Name}\\
% Second Institution
% \and
% {\rm Third Name}\\
% Third Institution
}

\maketitle

% Use the following at camera-ready time to suppress page numbers.
% Comment it out when you first submit the paper for review.
\thispagestyle{empty}

\subsection*{Abstract}
A quick understanding of the spread of Internet censorship has become important especially now that
many countries are using it to shutdown protests. It appears that controlling Internet traffic and
routing through different censorship techniques are easy to implement by governments and ISPs. While
researchers and activists are having a hard time find a distributed research volunteered clients to
investigate the state of accesicibality of Internet in different regions around the globe especially
for the ones that matter the most. In this paper, we discuss our efforts to find a proper/desired platform
to investigate CASESTUDY and CASESTUDY. We used RIPE ATLAS which is  (talk about benefits) but
(negatives). We propose an algorithm to balance timeliness, diversity , and cost to facilitate the
usage of ripe.

\section{Introduction}

Talking about the importance of RIPE Atlas in the past months. Refer to papers that used RIPE to collect data(\cite{lutu2014understanding} and \cite{brownlee2014searching}).

% Point out what problems the world is facing and how RIPE Atlas can help.

% Quick description of Atlas.
RIPE Atlas~\cite{atlas} is an Internet measurement network which was started in 2010 by RIPE NCC.
All measurement probes which constitute the network are run by volunteers.  Once a user connected
her probe to the network, it can be used for measurements and the user is also able to run custom
measurements by spending some of the network's currency, Atlas credits.  These credits are awarded
automatically based on the uptime of contributed probes.

% What kind of measurements does Atlas allow?
As of today, Atlas allows four types of measurements; ping, traceroute, DNS resolution, and X.509
certificate fetching.  All four measurement types can further be parameterised for more fine-grained
control.

% What are our contributions?  Is there more?
This work has three main contributions.
\begin{itemize}
	\item We evaluate the aptitude of RIPE Atlas for censorship analysis.
	\item We propose an algorithm to balance
	timeliness, diversity, and cost to facilitate effective censorship analysis.\footnote{All our
	code will be publicly available but we redacted the URL for anonymisation.}
	\item We present censorship analysis results based on several months of measurements.
\end{itemize}


\section{Related Work}
% This section will review three topics: first current existing direct network measurment tools such as
% Dimes, planet lab ,... Second Talks about other ways to do censorship analysis which is pretty much
% idle scanning and using side channels ... Third: we also need to cite recent papers/articles using
% Ripe for Internet measurment studies...
% TO DO: Philipp and Roya ...

% Longitudinal studies.
It is not difficult to conduct one-off censorship studies because censors typically do not have
sufficient time to react and thwart the study.  Longitudinal studies, on the other hand, are more
challenging as they have to be designed in a tamper-proof way.  In 2007, Crandall et al.  proposed
ConceptDoppler~\cite{Crandall2007}.  The design enables longitudinal censorship analysis by
detecting which keywords are filtered by the Great Firewall of China (GFW) over time.  More
recently, CensMon was introduced by Sfakianakis et al. in 2011~\cite{Sfakianakis2011}.  CensMon is a
web censorship monitor which is run on top of PlanetLab~\cite{planetlab}.  Most recently,
Filast\`{o} and Appelbaum presented OONI~\cite{Filasto2012} in 2012.  In contrast to CensMon and
ConceptDoppler, OONI is deployed and has been used successfully.\footnote{Gathered reports are
available online \url{https://ooni.torproject.org/reports/}.}  In parallel to all open platforms
discussed above, proprietary platforms exist~\cite{herdict,opennet} but we will not discuss them as
their closed nature does not encourage analysis.


\section{Measurment Cost}


\begin{table}[h] 

\hspace{2in}
\centering % centering table 
\begin{tabular}{c rr} % creating eight columns 
\hline\hline %inserting double-line 
Unit Measurement& Type \multicolumn{3}{c}{Cost} \\ [0.5ex] 
\hline % inserts single-line 
DNS\slash DNS6 & TCP & 20\\ % Entering row contents 
DNS\slash DNS6 & UDP & 10\\ 
SSLCert\slash SSLCert6 & & 10 \\
Ping\slash Ping6 & & $N * (int(S/1500)+1)$\\
Traceroute\slash Traceroute6 & & $ 10*N*(int(S/1500)+1)$\\
%%News & 9 & -3 & 7& 9& -5& -1& 9\\[1ex] % [1ex] adds vertical space 
\hline % inserts single-line 
\end{tabular} 
\caption{Measurement Cost Model where N and S are number of packet in the train and packet size, respectively.} %title of the table 
\label{tab:hresult} 
\end{table} 


\cite{Bortzmeyer14,Aben14}
\section{Discussion}

\subsection{Ethical Considerations}

% TODO Roya will fill this up later.
The problem: we might endanger probe contributors by running censored queries over their probes.
Contributors probably don't expect that their probe is used for censorship analysis.

Our justification: Plausible deniability, we are limited to X.509/DNS/ping/traceroute.  We are not
claiming that everything is OK.  Instead, we encourage further discussion.


\section*{Acknowledgments}

We would like to thank the anonymous reviewers for their constructive feedback on earlier versions of this paper. Additionally, we would like to thank Vesna Manojlovic for providing startup credit and the RIPE Atlas community for their support. Collin Anderson was supported by the University of Pennsylvania's Annenberg School for Communication.

\raggedright
\printbibliography

\theendnotes

\end{document}
