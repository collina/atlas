% TEMPLATE for Usenix papers, specifically to meet requirements of
%  USENIX '05
% originally a template for producing IEEE-format articles using LaTeX.
%   written by Matthew Ward, CS Department, Worcester Polytechnic Institute.
% adapted by David Beazley for his excellent SWIG paper in Proceedings,
%   Tcl 96
% turned into a smartass generic template by De Clarke, with thanks to
%   both the above pioneers
% use at your own risk.  Complaints to /dev/null.
% make it two column with no page numbering, default is 10 point

% Munged by Fred Douglis <douglis@research.att.com> 10/97 to separate
% the .sty file from the LaTeX source template, so that people can
% more easily include the .sty file into an existing document.  Also
% changed to more closely follow the style guidelines as represented
% by the Word sample file. 

% Note that since 2010, USENIX does not require endnotes. If you want
% foot of page notes, don't include the endnotes package in the 
% usepackage command, below.

% This version uses the latex2e styles, not the very ancient 2.09 stuff.
\documentclass[letterpaper,twocolumn,10pt]{article}
\usepackage{usenix,epsfig,endnotes}
\begin{document}

%don't want date printed
\date{}

%make title bold and 14 pt font (Latex default is non-bold, 16 pt)
\title{\Large \bf Our Paper}

%for single author (just remove % characters)
\author{
{\rm First Name}\\
First Institution
\and
{\rm Second Name}\\
Second Institution
\and
{\rm Third Name}\\
Third Institution
% copy the following lines to add more authors
% \and
% {\rm Name}\\
%Name Institution
} % end author

\maketitle

% Use the following at camera-ready time to suppress page numbers.
% Comment it out when you first submit the paper for review.
\thispagestyle{empty}

\subsection*{Abstract}
Your Abstract Text Goes Here.  Just a few facts.
Whet our appetites.

\section{Introduction}
%train of thoughts: talking about the recent popularity of censorship in the world and countires in
%africa and middle easts(cite news article) Then talk about the exsisting methods and briefly critic each onem then
%talk about what were the charactestics we were interested to have in current tools, then talk about
%Ripe atlas as a suitable tool, then talk about their issue related to cost function then talk about
%our contributions.

% talking about the scope of the problem
Tools for discovering censorship are essential considering their importance for journalists and
activists in many countires. Currently, the existaning infruastructures lack representetive(measurment machines) in
regions that they matter the most or they are not easy to use. 

Talking about the importance of RIPE Atlas in the past months. Refer to papers that used RIPE to collect data(\cite{lutu2014understanding} and \cite{brownlee2014searching}).

% Point out what problems the world is facing and how RIPE Atlas can help.

% Quick description of Atlas.
RIPE Atlas~\cite{atlas} is an Internet measurement network which was started in 2010 by RIPE NCC.
All measurement probes which constitute the network are run by volunteers.  Once a user connected
her probe to the network, it can be used for measurements and the user is also able to run custom
measurements by spending some of the network's currency, Atlas credits.  These credits are awarded
automatically based on the uptime of contributed probes.

% What kind of measurements does Atlas allow?
As of today, Atlas allows four types of measurements; ping, traceroute, DNS resolution, and X.509
certificate fetching.  All four measurement types can further be parameterised for more fine-grained
control.

% What are our contributions?  Is there more?
This work has three main contributions.
\begin{itemize}
	\item We evaluate the aptitude of RIPE Atlas for censorship analysis.
	\item We propose an algorithm to balance
	timeliness, diversity, and cost to facilitate effective censorship analysis.\footnote{All our
	code will be publicly available but we redacted the URL for anonymisation.}
	\item We present censorship analysis results based on several months of measurements.
\end{itemize}


\section{Related Work}
\label{related_work}
\begin{table*}[ht!]
\centering
\begin{tabular}{l|cccc}
\textbf{Platform} & \textbf{Flexibility} & \textbf{Coverage} &
\textbf{Blocking resistance} & \textbf{Main use} \\
\hline 
PlanetLab~\cite{planetlab} & High & Low/Medium & Medium & Network measurements \\
Atlas~\cite{atlas} & Low & Medium/High & Medium & Network measurements \\
M-Lab~\cite{dovrolis2010measurement} & Low & High & Medium & Network measurements \\
Tor~\cite{Dingledine2004} & Medium & Medium & Low & Low-latency anonymity \\
OONI~\cite{Filasto2012} & High & Low & Medium & Censorship analysis \\
Herdict~\cite{Herdict} & Low & Low/Medium & Low & Censorship analysis \\
OpenNet~\cite{opennet} & Low & Medium & High & Censorship analysis \\
\hline 
\end{tabular} 
\caption{Comparison between several popular censorship analysis platforms.}
\label{tab:comparison}
\end{table*}

% Longitudinal studies.
It is not difficult to conduct one-off censorship studies because censors
typically do not have sufficient time to react and thwart the research.
Longitudinal studies, on the other hand, are more challenging as they have to
be designed in a tamper-proof and sustainable way.  In 2007, Crandall et al.
proposed ConceptDoppler~\cite{Crandall2007}.  The design enables longitudinal
censorship analysis by detecting which keywords are filtered by the Great
Firewall of China (GFW) over time.  More recently, CensMon was introduced by
Sfakianakis et al. in 2011~\cite{Sfakianakis2011}.  CensMon is a web censorship
monitor which is run on top of PlanetLab~\cite{planetlab}.  Most recently, in
2012, Filast\`{o} and Appelbaum presented OONI~\cite{Filasto2012}.  In
contrast to CensMon and ConceptDoppler, OONI is deployed and has been used
successfully.\footnote{Gathered reports are available online at:
\url{https://ooni.torproject.org/reports/}.}  In parallel to the open platforms
discussed above, proprietary platforms exist~\cite{hwang2007herdict,opennet}.


% Comparison to other projects.
Table~\ref{tab:comparison} contains a comparison between popular and deployed
platforms which are or can be used for censorship analysis.  Our comparison is
based on \emph{flexibility} (i.e., how many types of measurements can be run),
\emph{coverage} (i.e., how many probes in how many countries are available),
and \emph{blocking resistance} (i.e., how easy it is for censors to disable the
respective platform).  We qualitatively compare all platforms and assign them
the labels ``Low'', ``Medium'', or ``High''.  Note that we do not propose
Atlas as \emph{replacement} for any existing censorship measurement platforms.
Instead, we see it as a \emph{complement} that contributes to the already
existing and growing landscape of initiatives.

% Side channel measurements.
% TODO - Are there more papers we are missing?
Additionally, in the absence of deployed platforms or other means to access machines inside
censoring countries, censorship analysts have resorted to exploiting TCP/IP
side channels.  In particular, Ensafi et al. demonstrated that it is possible
to measure intentional packet dropping without controlling either the source or
the destination machine~\cite{Ensafi2014}.

% Previous Atlas censorship analysis.
Atlas has already been used as platform for censorship analysis outside an
academic setting.  In 2014, Maass used Atlas to find inconsistencies in the DNS
records and X.509 certificates for torproject.org~\cite{Maass2014}.  In the
same year, Bortzmeyer and Aben independently discussed Internet censorship in
Turkey~\cite{Bortzmeyer14,Aben14}.  While we discuss the same topic in
Section~\ref{sec:case_studies}, we do so with significantly more data and in a
more rigorous fashion based on the privilege of time.

\section*{Acknowledgments}

We would like to thank the anonymous reviewers for their constructive
feedback on earlier versions of this paper. Additionally, we would like
to thank Vesna Manojlovic for providing startup credit and the RIPE
Atlas community for their insight. Collin Anderson was supported by the
Internet Policy Observatory program at the University of Pennsylvania's
Annenberg School for Communication.

{\footnotesize \bibliographystyle{acm}
\bibliography{bibliography}}

\theendnotes

\end{document}
