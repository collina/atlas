% TEMPLATE for Usenix papers, specifically to meet requirements of
%  USENIX '05
% originally a template for producing IEEE-format articles using LaTeX.
%   written by Matthew Ward, CS Department, Worcester Polytechnic Institute.
% adapted by David Beazley for his excellent SWIG paper in Proceedings,
%   Tcl 96
% turned into a smartass generic template by De Clarke, with thanks to
%   both the above pioneers
% use at your own risk.  Complaints to /dev/null.
% make it two column with no page numbering, default is 10 point

% Munged by Fred Douglis <douglis@research.att.com> 10/97 to separate
% the .sty file from the LaTeX source template, so that people can
% more easily include the .sty file into an existing document.  Also
% changed to more closely follow the style guidelines as represented
% by the Word sample file. 

% Note that since 2010, USENIX does not require endnotes. If you want
% foot of page notes, don't include the endnotes package in the 
% usepackage command, below.

% This version uses the latex2e styles, not the very ancient 2.09 stuff.
\documentclass[letterpaper,twocolumn,10pt]{article}

\usepackage{usenix,epsfig,endnotes}
\usepackage{flushend}
\usepackage[draft]{hyperref}
\usepackage{color}
\usepackage[backend=bibtex,sortcites]{biblatex}
\bibliography{bibliography}
\usepackage{multirow}

% Don't use monospace URLs.
\urlstyle{sf}

\definecolor{dred}{rgb}{0.8,0,0}
\hypersetup{
	pdftitle={Global Network Interference over the RIPE Atlas Network},
	pdfauthor={Collin Anderson, Philipp Winter, Roya},
	pdfkeywords={ripe atlas, censorship, filtering, network interference, measurement},
	colorlinks=true,
	urlcolor=blue,
	linkcolor=blue,
	citecolor=blue
}

\begin{document}

%don't want date printed
\date{}

%make title bold and 14 pt font (Latex default is non-bold, 16 pt)
\title{
	\Large \bf Global Network Interference Detection \\
	over the RIPE Atlas Network
}

\author{
	{\rm Collin Anderson} \\
	University of Pennsylvania
	\and
	{\rm Philipp Winter} \\
	Karlstad University
	\and
	{\rm Roya} \\
	Independent Researcher
}

\maketitle

% Use the following at camera-ready time to suppress page numbers.
% Comment it out when you first submit the paper for review.
\thispagestyle{empty}

\subsection*{Abstract}

As Internet filtering practices proliferate internationally, researchers can no longer presume the consistent and homogenous imposition of information controls, evoking the need for pervasive observation longitudinally across national networks. Existing interference measurement platforms frequently suffer from poor adoption, insufficient geographic coverage, and scalability problems. Therefore, in order to outline an analytical framework and data collection needs for future ubiquitous measurements initiatives, we build on top of the existent and widely-deployed RIPE Atlas platform. We propose methods for monitoring the reachability of vital services through an algorithm that balance timeliness, diversity, and cost. We then use Atlas to investigate blocking events in Turkey and Russia. Our measurements uncover previously under-examined forms of interference and evidence of cooperation between a well-known blogging platform and government authorities for purposes of blocking hosted content.

\section{Introduction}

Talking about the importance of RIPE Atlas in the past months. Refer to papers that used RIPE to collect data(\cite{lutu2014understanding} and \cite{brownlee2014searching}).

% Point out what problems the world is facing and how RIPE Atlas can help.

% Quick description of Atlas.
RIPE Atlas~\cite{atlas} is an Internet measurement network which was started in 2010 by RIPE NCC.
All measurement probes which constitute the network are run by volunteers.  Once a user connected
her probe to the network, it can be used for measurements and the user is also able to run custom
measurements by spending some of the network's currency, Atlas credits.  These credits are awarded
automatically based on the uptime of contributed probes.

% What kind of measurements does Atlas allow?
As of today, Atlas allows four types of measurements; ping, traceroute, DNS resolution, and X.509
certificate fetching.  All four measurement types can further be parameterised for more fine-grained
control.

% What are our contributions?  Is there more?
This work has three main contributions.
\begin{itemize}
	\item We evaluate the aptitude of RIPE Atlas for censorship analysis.
	\item We propose an algorithm to balance
	timeliness, diversity, and cost to facilitate effective censorship analysis.\footnote{All our
	code will be publicly available but we redacted the URL for anonymisation.}
	\item We present censorship analysis results based on several months of measurements.
\end{itemize}


\section{Related Work}
% This section will review three topics: first current existing direct network measurment tools such as
% Dimes, planet lab ,... Second Talks about other ways to do censorship analysis which is pretty much
% idle scanning and using side channels ... Third: we also need to cite recent papers/articles using
% Ripe for Internet measurment studies...
% TO DO: Philipp and Roya ...

% Longitudinal studies.
It is not difficult to conduct one-off censorship studies because censors typically do not have
sufficient time to react and thwart the study.  Longitudinal studies, on the other hand, are more
challenging as they have to be designed in a tamper-proof way.  In 2007, Crandall et al.  proposed
ConceptDoppler~\cite{Crandall2007}.  The design enables longitudinal censorship analysis by
detecting which keywords are filtered by the Great Firewall of China (GFW) over time.  More
recently, CensMon was introduced by Sfakianakis et al. in 2011~\cite{Sfakianakis2011}.  CensMon is a
web censorship monitor which is run on top of PlanetLab~\cite{planetlab}.  Most recently,
Filast\`{o} and Appelbaum presented OONI~\cite{Filasto2012} in 2012.  In contrast to CensMon and
ConceptDoppler, OONI is deployed and has been used successfully.\footnote{Gathered reports are
available online \url{https://ooni.torproject.org/reports/}.}  In parallel to all open platforms
discussed above, proprietary platforms exist~\cite{herdict,opennet} but we will not discuss them as
their closed nature does not encourage analysis.


\section{Framework Structure}
\label{sec:framework}
We now discuss Atlas' aptitude as censorship analysis platform and continue by
presenting \textsf{Cartography}, our censorship analysis framework which is
based on RIPE Atlas.

\subsection{RIPE Atlas Background}
% Some basic information about Atlas.
Having been founded in 2010 by RIPE NCC, Atlas~\cite{atlas} is a globally
distributed Internet measurement network consisting of probes run by
volunteers.  Once a user connected her probe to the network, it can be used for
measurements and the user is also able to run custom measurements by spending
some of the network's currency, Atlas credits.  These credits are awarded
automatically based on the uptime of contributed probes.  Measurements can be
run either over the web frontend, or over a HTTP-based API.

% Geographic and topological distribution.
An ideal censorship measurement platform features high geographic and
topological diversity, thereby facilitating measurements in any region where
censorship occurs.  While Atlas probes are distributed throughout the world,
there is a significant bias towards the U.S. and Europe as can be seen in
Figure~\ref{fig:probe_distribution}.  As for Atlas' topography, only 68
autonomous systems contain 40\% of all Atlas probes with the three most common
autonomous systems being 7922 (4.4\%, Comcast Cable Communications), 3320
(3.2\%, Deutsche Telekom), and 6830 (2.8\%, Liberty Global Operations).  While
not optimal, most censoring regions still contain at least several probes.

\begin{figure}[t]
\centering
\includegraphics[width=0.48\textwidth]{diagrams/probe_distribution.jpg}
\caption{The geographic distribution of Atlas probes as of May 2014.  Green
icons represent active probes and red icons represent probes which are
currently offline.  The distribution is biased towards the U.S. and Europe.}
\label{fig:probe_distribution}
\end{figure}

% What kind of measurements does Atlas allow?
As of May 2014, Atlas allows four types of measurements; ping, traceroute, DNS
resolution, and X.509 certificate fetching.  All four measurement types can
further be parameterized for more fine-grained control.  HTTP requests are not
possible at this point due to concerns.  While Atlas clearly lacks the
flexibility of comparable platforms (see Table~\ref{tab:comparison}), it makes
up for it with comparably high diversity and its continued growth.  After all,
we do not expect Atlas to replace existing platforms such as OONI but rather to
\emph{complement} them.

% TODO - Should we talk about our crappy command-line interface?

% We also need to mention how you can stablish a big experiments using RIPE.
% And mention about the propblem of how you frist need to submit your
% experiments and then it will give you predicted cost. This way we can move to
% cost function.
As briefly mentioned above, Atlas' measurement have to be paid with so-called
credits.  The exact ``price'' of a measurement depends on the measurement type,
its parameters, and the destination(s).

\subsection{Ethical Aspects}
% The problem.
Atlas was not designed as censorship analysis platform and accordingly, its
volunteers likely do not expect that their probes will be used for such
purposes.  Careless measurements could attract a censor's attention and cause
repercussions for the respective probe operator.

% How we justify our research.
Recall that Atlas' measurement types are limited to ping, traceroute, DNS
requests and X.509 certificate fetching.  As of May 2014, it is not possible to
create HTTP requests or engage in actual, meaningful communication with
arbitrary destinations which limits the damage caused by reckless measurement.
Nevertheless, we acknowledge that care must always be taken and hope to initiate
further ethical discussions.


\section{Case studies}

% here we need to add the turkey example. Collin you can filled this part. Philipp I remember you talked about the Tor stuff. Please don't do anything new.
% instead spend time collect and analyse what we have.


\section{Conclusion}
\label{sec:conclusion}

In this paper, we evaluated the RIPE Atlas platform for censorship analysis.
We

Our measurements and analysis provide an early perspective on the opportunities
and methodologies possible with pervasive censorship research. Previous
examinations of Internet filtering have tended to analyze specific apparatuses
on a per-country basis, assuming internal consistency. This past approach has
been appropriate for describing the diversity of methods used to control access
globally, as well as for when the primary research focus is on content themes
and in countries that impose restrictions at central points of transit. 

However, as Internet censorship has increased in countries and polities with
competition and private markets at the international frontier, researchers can
no longer assume direct and homogenous control by authorities. Our applications
of Cartography against two recent and large-scale censorship incidents in
Russia and Turkey demonstrate this. Russia and Turkey's networks are more
administratively and technically decentralized than China and
Iran~\cite{Roberts2011}. Delays or irregulatories of implementation sheds light
on the inconsistences of the administration of censorship. 

Our initial research demonstrates that across national networks there are
substantive differences of methods, rates of implementation, and, in at least
one case, even selective compliance for content restrictions. Moreover, the
available measurements of Atlas proved sufficient for Cartography to uncover
interesting and evolving censorship practices in jurisdictions that are only
beginning to assert expansive control of online content.

Our code is available online at \url{http://cartography.nymity.ch}.


\section*{Acknowledgments}

We would like to thank the anonymous reviewers for their constructive feedback on earlier versions of this paper. Additionally, we would like to thank Vesna Manojlovic for providing startup credit and the RIPE Atlas community for their support. Collin Anderson was supported by the University of Pennsylvania's Annenberg School for Communication.

\raggedright
\printbibliography

% \theendnotes

\end{document}
