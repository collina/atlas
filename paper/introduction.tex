\section{Introduction}
%train of thoughts: talking about the recent popularity of censorship in the world and countires in
%africa and middle easts(cite news article) Then talk about the exsisting methods and briefly critic each onem then
%talk about what were the charactestics we were interested to have in current tools, then talk about
%Ripe atlas as a suitable tool, then talk about their issue related to cost function then talk about
%our contributions.

%TODO: Collin and Philipp find/cite some news about african countries adopting censorship softwares.

% talking about the scope of the problem
Tools for discovering censorship are essential considering their importance for journalists and
activists in many countires. Currently, the existaning infruastructures lack representetive(measurment machines) in
regions that they matter the most or they are not easy to use. 

Talking about the importance of RIPE Atlas in the past months. Refer to papers that used RIPE to collect data(\cite{lutu2014understanding} and \cite{brownlee2014searching}).

% Point out what problems the world is facing and how RIPE Atlas can help.

% Quick description of Atlas.
RIPE Atlas~\cite{atlas} is an Internet measurement network which was started in 2010 by RIPE NCC.
All measurement probes which constitute the network are run by volunteers.  Once a user connected
her probe to the network, it can be used for measurements and the user is also able to run custom
measurements by spending some of the network's currency, Atlas credits.  These credits are awarded
automatically based on the uptime of contributed probes.

% What kind of measurements does Atlas allow?
As of today, Atlas allows four types of measurements; ping, traceroute, DNS resolution, and X.509
certificate fetching.  All four measurement types can further be parameterised for more fine-grained
control.

% What are our contributions?  Is there more?
This work has three main contributions.
\begin{itemize}
	\item We evaluate the aptitude of RIPE Atlas for censorship analysis.
	\item We propose an algorithm to balance
	timeliness, diversity, and cost to facilitate effective censorship analysis.\footnote{All our
	code will be publicly available but we redacted the URL for anonymisation.}
	\item We present censorship analysis results based on several months of measurements.
\end{itemize}
