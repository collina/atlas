\section{Introduction}
%train of thoughts: talking about the recent popularity of censorship in the world and countires in
%africa and middle easts(cite news article) Then talk about the exsisting methods and briefly critic
%each one them
%talk about what were the charactestics we were interested to have in current tools, then talk about
%Ripe atlas as a suitable tool, then talk about their issue related to cost function then talk about
%our contributions.
Two problematic issues are currently happening regards of Internet state: on one hand tools to apply
censorship(in large scale like countries) is more available. Iranian government or African countries
are just buying tools from China ... on the other hand it is challenging to find a reliable measurement machines.

There are mainly two different ways to detect blockages of websites. Researchers can directly
investigate conncetivity of clients in different countries by using volunteered distributed
infrastructures such as PlanetLab(Cite), M-lab(Cite),and DIMES(Cite). As shwon by many previous works(cites),beside that the number of
nodes are geographically and topologically biased, deploying any measurements take whiles.
Researchers can also indirectly check the intentional packet drops using different side channels
such as idle scans(cite roya's paper). The limitation with that approach is that current state of
the arts for the idle scans are limited to test layer three connectivity. 
%TODO: Collin and Philipp find/cite some news about african countries adopting censorship softwares.
In the process of finding a proper tool to deploy quickly to test censorship, we summerized the
properties that the tool should have. In our opinion, these are the necessary properties:
availability, easy to use,  .  After analysing
different existing tools we agreed that RIPE ATLAS is actually a good choice to do internet
measurements studies. There are easy to setup and easy to use. The atlas for interesting regions  
Talking about the importance of RIPE Atlas in the past months.

RIPE Atlas~\cite{atlas} is an Internet measurement network which was started in 2010 by RIPE NCC.
All measurement probes which constitute the network are run by volunteers.  Once a user connected
her probe to the network, it can be used for measurements and the user is also able to run custom
measurements by spending some of the network's currency, Atlas credits.  These credits are awarded
automatically based on the upme of contributed probes.

% What kind of measurements does Atlas allow?
As of today, Atlas allows four types of measurements; ping, traceroute, DNS resolution, and X.509
certificate fetching.  All four measurement types can further be parameterised for more fine-grained
control.

% What are our contributions?  Is there more?
This work has three main contributions.
\begin{itemize}
	\item We evaluate the aptitude of RIPE Atlas for censorship analysis.
	\item We propose an algorithm to balance
	timeliness, diversity, and cost to facilitate effective censorship analysis.\footnote{All our
	code will be publicly available but we redacted the URL for anonymisation.}
	\item We present censorship analysis results based on several months of measurements.
\end{itemize}
