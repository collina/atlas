\section{Introduction}
Aided by network equipment vendors in the United States, Europe, and China,
censoring regimes are tightening their grip on national communication
infrastructure and continue to choke the Internet.  An important counter
strategy is to measure, document, and expose Internet censorship incidents.
Sunlight is known to be a disinfectant and by shedding light on these events,
the public's attention can be drawn toward these actions which are frequently
unlawful and unjust.  Several methods exist to measure censorship.  Ideally,
the analyst has control over a censored source and an uncensored destination
computer.  This is typically not the case, so research has to resort to open
proxies, the help of volunteers, and existing censorship measurement platforms.

All these methods have disadvantages.  Open proxies provide suffer from low
network coverage, are unreliable, and are often limited to TCP streams or HTTP
requests.  Cooperation with volunteers can lead to repurcissions and is time
consuming.  Finally, censorship measurement platforms also suffer from limited
network coverage and insufficient maintenance.  While are not able to solve all
of these problems, we build our analysis platform on top of an existing, widely
deployed and successful network measurement platform, RIPE's Atlas.  That way,
we can largely ignore factors such as maintenance and outreach as it is taken
care of by the network operator.  Instead, we only focus on making use of the
existing resources in an effective way.

Finally, we believe that censorship analysis platforms do not have to compete
with each other.  Indeed, existing platforms such as
PlanetLab~\cite{planetlab}, Herdict~\cite{Herdict}, and OONI~\cite{Filasto2012}
complement each other and provide analysts with diverse and unique views on
Internet censorship.  We believe that a censorship analysis platform based on
Atlas can provide an additional angle to the big picture whose strengths are
wide deployment and rapid results as we will show later in this paper.

\begin{itemize}
	\item We evaluate the aptitude of RIPE's Atlas platform for censorship
		analysis and propose an algorithm to balance timeliness, network
		diversity, and cost to facilitate effective censorship analysis.
	\item We apply the platform and algorithm for analysis of ongoing
		censorship events across different countries, and results based on
		several months of measurements.
\end{itemize}

The remainder of this paper is structured as follows.
Section~\ref{related_work} begins by giving an overview of related work which
is then followed by our \textsf{Cartography}'s framework structure in
Section~\ref{sec:framework}.  Next, we present two case studies in
Section~\ref{sec:case_studies} and conclude the paper
in~Section~\ref{sec:conclusion}.
