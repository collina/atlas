\section{Case studies}

% *** Can I try to negotiate with all dir authorities?

\subsection{State Responses to Mass Circumvention of Twitter Ban in Turkey}

Beginning March 20, 2014, social media users began to report limitations on the availability of Twitter across the Turkey's Internet Service Providers. This escalation of content restrictions had been foreshadowed by news and statements from political figures over the previous two months. YouTube and Twitter in name had both become increasingly the target of condemnation by Prime Minister Erdogan, stemming from their use to distribute audio and documents alleging financial and political impropriety by public officials ahead of local elections. Based on outstanding court complaints ordering the removal of content and accounts deemed defamatory, the Turkish government's Information and Communication Technologies Authority (BTK) mandated the filtering of Twitter on national ISP. Turkey's Internet filtering has previously been connected to DNS tampering and IP blocking \cite{akdeniz2010report}, which both fall under the measurements possible through Atlas. Given both the rhetoric preceding the ban, and the public's reaction in response, upon news of the Twitter ban, periodic measurements were queued for a broad set of predictions about potential next steps by authorities. Through the ten probes covering nine ASNs, we scheduled hourly measurements of local DNS answers, SSL connectivity, and Traceroute reachability for Twitter, YouTube, Google Public DNS and the Tor Project.

\begin{figure*}
  \includegraphics[width=\textwidth]{resources/tr-20140321-20140407-social_media_filtering.png}
  \label{image:tr-social_media_filtering}
  \caption{Disruption of Social Media Platforms in Turkey, March - April 2014}
\end{figure*}

As illustrated in Figure \ref{image:tr-social_media_filtering}, we find a rapid progression on the part of Turkey's authorities and telecommunications companies that compromises of at least six shifts in content restrictions and blocking strategies within a two week period.

While the BTK and compliant ISPs rely on local DNS manipulation and IP blocking in order to enforce content restriction, it appears that the former is more commonly used by authorities. As of April 24, 2014, the Turkish-language anti-censorship site \textit{Engelliweb}, which tracks addresses of blocked content, only lists 167 IP addresses restricted in country, compared to 40566 domain names \cite{engelliweb}. The popularity of content distribution networks and shared hosting easily makes address-based blocking either overbroad, creating collateral damage from restricting all domains and content on a shared host, or inefficiently narrow. Iranian authorities reportedly faced an aspect of this in October 2012, when an attempt to block  YouTube was blamed for week-long disruptions on access to Gmail, likely a result of shared infrastructure and addresses across Google's services \cite{bbc2012gmail}. In absence of address blocking or HTTP filtering, users that received valid DNS answers for Twitter's domain names could browse without further interference. As a result, foreign DNS servers quickly became both a circumvention mechanism and a political statement, with the addresses of alternative services offered by Google and OpenDNS reportedly graffitied across the the country in protest of the ban.

The ease and rate that users were appeared to be bypassing the filters, by some indications more tweets were sent after the restrictions than before, prompted broader crackdown by authorities. On the morning of March 22nd (\textbf{Figure \ref{image:tr-social_media_filtering}, Event A}), between 01:00 and 02:00, backbone providers Tellcom Iletisim Hizmetleri and Turk Telekom began disrupting Google's Public DNS service through the IP blocking of its two prominent addresses (8.8.8.8 and 8.8.4.4). This restriction would have disrupted access to filtered and non-filtered sites alike through interfering with the translation of names to Internet-routable addresses. Use of Google's DNS is commonplace in the country, and not necessarily indicative of an intent to circumvent filters. By some metrics 18.3\% of Internet users in Turkey rely on Google Public DNS \cite{ispcolumn2013googledns}. Given this cost, within hours, by 6:00 the same morning, the DNS blocking had been removed across all ISPs. Instead, to buttress the restrictions, providers shortly began to drop all outgoing traffic to addresses associated with the twitter.com domain, regardless of port, protocol or provider (\textbf{Event B}). By 16:00 of that day, no Atlas probe could directly negotiate an SSL connection with Twitter until the removal of the ban nearly two weeks later.

On March 27 (\textbf{Event C}), after recordings were posted of Turkish national security officials discussing possible military action against Syria, YouTube was blocked through false answers to DNS for the youtube.com domain. Within the Atlas network, this restriction appears as slow decline in the number of probes able to establish a connection to the media platform, implying gradual compliance with an administrative order. However, unlike Twitter, a significant minority of probes remained able to communicate with YouTube. Google's infrastructure presents substantial risk of collateral damage that could result from address or network prefix restrictions, which were not present with Twitter, and thus clients that were able to receive a valid address could reliably bypass the ban.

Although the BTK is alleged to have ordered the installation of deep packet inspection equipment for purposes of lawful interception and censorship \cite{kirlidog2011deep}, restrictions on circumvention and anonymizations tools appear to have been limited to the filtering of services' websites. Beginning March 28th 19:00, Atlas probes in-country began fail to establish an SSL connections to torproject.org (\textbf{Event D}). However, this restriction neither included IP restrictions, nor was there evidence of interference with the accessibility of the network. Atlas probes could continue to negotiate valid connections to Tor's Directory Authories. Throughout the increased manipulation of local DNS services, nearly half of the Atlas probes remained connected due to their use of foreign DNS services. More aggressive restrictions were perhaps inevitable due to international attention on the rapid adoption of circumvention tools and Google Public DNS as an indicator of the futility of the ban.

\begin{figure*}
  \includegraphics[width=\textwidth]{resources/traceroute-TR-google_public_dns-20140330-hijack.png}
  \label{image:tr-ttnet_hijack}
  \caption{Turkish Telecom (AS9121) Hijack of Google Public DNS Traffic, March 30 2014}
\end{figure*}

On the evening of March 28th, hosts querying foreign-based DNS servers began to receive the same false answers as those provided domestically, leading to a rapid drop in reported availability of YouTube and Tor (\textbf{Event E}). A publicly-available traceroute measurement scheduled on the Atlas network by third-parties against Google's Public DNS returned idiosyncratic and spontaneous shifts in Turkey's network topology timed in relation to these changes. These changes appeared within traceroutes as a shortening in the number of hops to Google with a multifold reduction in traffic latency and the absence of international hosts in path. The core telecommunications provider Turk Telekom had begun to reroute traffic destined for Google to a local DNS server, Figure \ref{image:tr-ttnet_hijack}. Only TEKNOTEL Telekom maintained consistently valid routes for Google Public DNS, through Telecom Italia Sparkle, however, two days later Doruk \.{I}leti\c{s}im and Net Elektronik Tasar{\i}m reestablished connectivity through Euroweb Romania. Turk Telekom's redirection was finally removed late on April 7.

By April 3th, despite continued hijacking of Google's Public DNS for filtering purposes and interference with YouTube, Twitter was unblocked for all probes (\textbf{Event F}).

\subsection{Private Sector Cooperation in Russian Censorship of Alexei Navalny}

Concurrent to broader restrictions on independent media, on March 13, 2014, Russia's Federal Service for Supervision in the Sphere of Telecom, Information Technologies and Mass Communications (Roskomnadzor) announced the blacklisting of opposition figure Alexei Navalny's LiveJournal blog, due to claimed violations of the terms of his house arrest. Shortly after the order was issued, Russian social media and state news agencies reported that access to the entirity of LiveJournal had been temporarily disrupted on some Interent providers due to their inability to distinguish traffic to different blogs within the site \cite{itar2014livejournal}. Overblocking has proven a recurrent issue in Russia. When a regional ISP, Netis Telekom, previously blocked the full site while complying with a court order against a neo-nazi blog, LiveJournal's Russian Director Ilya Dronov attributed the event to several ISPs' reliance on IP-based filtering and the inclusion of IP addresses within court orders \cite{leta2012livejoural}. 

At the same time as Navalny's censorship, four opposition news portals were filtered based on the allegations that they called for ``illegal activity and participation in mass events that are conducted contrary to the established order,'' including grani.ru \cite{ibtimes2014russia}. As with Turkey, Russia's content restrictions have previously been attributed to DNS poisoning and IP filtering, presenting the same potential circumvention strategies and measurement opportunities \cite{rugovdns, verkamp2012inferring}. However, with a random sample of 255 probes across 147 ASNs in Russia, only 38 probes on 20 ASNs received aberrant DNS answers. Within this subset, probes received a diverse, consistent selection of ten unique addresses, including two within RFC1918, private network address space (10.52.34.222 and 192.168.103.162). A greater selection, 40 probes across 23 ASNs, of traceroutes to the port 80 for the primary address associated with Grani as of April 30 (23.253.120.92) failed within Russia network space. Based on Russian filtering documentation efforts, the order that restricted Grani identifies at least fourteen addresses connected with the site \cite{antizapret2014}. An additional address that appears in the Grani order was found blocked on 36 probes and 22 ASNs, highlighting inconsistency in implementation and upkeep.

In contrast to Grani, a locally resolved DNS query for navalny.livejournal.com over 255 probes on 146 ASNs received a consistent reply of 208.93.0.190, which matched answers regionally with only one anomalous response, a formerly valid address. The blocking of Navalny's blog must be different from Grani. While the returned DNS A record of 208.93.0.190 falls within a network prefix owned by LiveJournal Inc. (208.93.0.0/22), within the 1462 LiveJournal subdomains in Alexa's Top 1 million list, 1450 blogs resolved another address, 208.93.0.150. Both hosts appear to front servers for the LiveJournal platform, as they return the same SSL Certificate and host the same content. Requests to 208.93.0.150 with a HTTP Host header set to navalny.livejournal.com retrieves the correct content and non-blacklisted content is retrievable through 208.93.0.190.

As of April 2014, only five subdomains on livejournal.com could be found whose DNS A records resolved to the address 208.93.0.190, Figure \ref{lj-blocked-blogs}, four of which are listed within Alexa's top sites. All the blogs found on this alternative host have been publicly declared by Russian authorities as in violation the country's media laws for promotion of political activities or extremism, and two are listed within publicly-available filter site lists. 

\begin{figure*}
  \includegraphics[width=\textwidth]{resources/atlas_cache-results-measurement_id-1663748.png}
  \label{image:ru-grani-hijack}
  \caption{Rostelecom (AS12389) Hijack of grani.ru Traffic, April 30 2014}
\end{figure*}

Based on timing, publicly-available history, available domain names records, and Atlas network measurements, it appears that a host was specially established to faciliate Russian censorship of the LiveJournal platform, potentially based on interest by the service to mitigate further overblocking of the platform. Using HTTPS Ecosystem Scans as a metric of accessibility \cite{projectsonar}, the LiveJournal frontend at 208.93.0.190 came online between February 10 and February 17, with the address otherwise unused until then. Two months later, the Ukrainian LiveJournal blog `Pauluskp' (pauluskp.livejournal.com), which had covered Russian involvement in Crimea, was filtered with the administrative order listing an IP Address of 208.93.0.190. However, as recently as six days before, the blog was recorded as pointing to the main LiveJournal host. Similarly, the movement of Navalny's blog was noticed within social media \ref{miptru2014}. It appears that in the lead up to or at the time of filtering orders, LiveJournal coordinates with authorities to alter the DNS A record for blogs designated by Roskomnadzor, in order to segregate blacklisted content from the rest of the platform [6].

\begin{table}
    \begin{tabular}{| l | c | c |}
        \hline
        Subdomain & Language & Roskomnadzor\\
        \hline
        drugoi-nnover & Russian & Yes\\
        m-athanasios & Russian & Yes\\
        imperialcommiss & Russian & Yes\\
        pauluskp & Russian & Yes \\
        navalny & Russian & Yes \\
        \hline
    \end{tabular}
    \label{table:lj-blocked-blogs}
    \caption{LiveJournal DNS A Records of 208.93.0.190}
\end{table}

Segregated LiveJournal content and a blacklisted addresses are subject to an additional, unknown method of network-layer interception performed within the backbone network of Rostelecom (AS12389). While LiveJournal blog content is not accessible over HTTPS, frontend hosts offer SSL services for the purpose of securing the transmission of user credentials. As of April 28, 255 of 343 probes within Russia were able to retrieve a valid and consistent LiveJournal SSL Certificate from the alternative LiveJournal host by address. Another 78 probes returned either irregular responses or failed to connect, of which 40 probes on 29 ASNs returned SSL certificates with common name or locations fields attributed to Russian ISPs. Based on HTTPS data, the four aberrant certificates captured have been seen previously on seven Russian addresses belonging to the State Institute of Information Technologies, Rostelecom and Electron telecom network. Three of these hosts are responsive and still match certificates, two are generic ISP homepages and one notifies of the blocking of the site `rutracker.ru.' Other measurements that are unresponsive could be indicative of port blocking by other intermediaries or the redirection of traffic to a server that is not listening for SSL connections.

The invalid certificates indicate that an intermediary in transit has intercepted or redirected the traffic out of its normal path to a third-party server controlled by Russian entities. This approach is different from the normal man-in-the-middle approach seen in countries such as Iran and Syria, and ellucidates the potential for Russian ISPs to falsify content or gather user credentials. This behavior is not limited to protocol or port, although the end host appears to be only responsive to TCP requests, Figure \ref{image:ru-grani-hijack}. Hollistic interference suggesting the redirection at the network layer, rather than application-based decisions or termination of traffic associated with deep packet inspection. Moreover, adjacent addresses within the same network and others destinations, such as the normal frontend for LiveJournal traverse a valid international path. Instead, blacklisted traffic appears to be coerced into a path controlled by Rostelecom that is otherwise not taking for destinations within the same network, indicating a narrowly-crafted interference with normal routing through false advertisements or forwarding.
