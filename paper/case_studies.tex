\section{Case studies}
\label{sec:case_studies}

After having presented our measurement platform and parameters, built on
top of Atlas, we now evaluate it by discussing two cases of large-scale
restrictions to online content and social media.  In particular,
Turkey's ban on media platforms in Section~\ref{sec:turkey} and Russia's
filtering of opposition LiveJournal content in Section~\ref{sec:russia}.
All dates and times reported follow Coordinated Universal Time.

% *** Can I try to negotiate with all dir authorities?

\subsection{Turkey's ban of Twitter}
\label{sec:turkey}

In late March, social media users began to report limitations on the
availability of Twitter across Turkey's Internet Service Providers.
YouTube and Twitter had both become the target of condemnation by Prime
Minister Recep Tayyip Erdo\u{g}an in preceding months. By March 20, the
Turkish government's Information and Communication Technologies
Authority (BTK) mandated the filtering of Twitter across the country's
service providers.

Turkey's Internet filtering has previously been characterized as DNS
tampering and IP blocking \cite{akdeniz2010report}, which both fall
under the measurements possible through Atlas.  Upon news of the Twitter
ban, we scheduled hourly measurements of local DNS answers, SSL
connectivity, and traceroute reachability for Twitter, YouTube, Google
Public DNS and the Tor Project through ten probes, covering nine ASNs. 
The selected measurement targets sought to longitudinally document the
Turkish government's disruption of controversial political content,
identified based statements by authorities and potential use for
circumventing controls. Seeking to address an immediate interest for
real-time awareness, the measurements did not attempt to assess the
whole of the country's content restrictions. As illustrated in
Figure~\ref{image:tr-social_media_filtering}, we found at least six
shifts in content restrictions and blocking strategies within a two week
period.

\begin{figure*}
  \centering
  \includegraphics[width=0.9\textwidth]{diagrams/tr-20140321-20140407-social_media_filtering.png}
  \caption{Disruption of Social Media Platforms in Turkey, March -- April 2014}
  \label{image:tr-social_media_filtering}
\end{figure*}


While the BTK and compliant ISPs rely on DNS manipulation and IP
blocking, it appears that the former is more popular.  As of April 24,
2014, the Turkish-language anti-censorship site Engelliweb~\cite{Engelliweb},
which tracks blocked content, only lists 167 IP addresses restricted in
country, compared to 40,566 domain names. In absence
of address blocking or HTTP filtering, users that received valid DNS
answers for Twitter's domain names could browse without further
interference. As a result, foreign DNS servers quickly became both a
circumvention mechanism and a political statement, with the addresses of
alternative services offered by Google and OpenDNS reportedly graffitied
across the the country in protest of the ban.

On the morning of March 22 (see
Figure~\ref{image:tr-social_media_filtering}, \textbf{Event A}), between
01:00 and 02:00, backbone providers Tellcom \.{I}leti\c{s}im Hizmetleri
and T\"{u}rk Telekom began disrupting Google Public DNS service through
the IP blocking of its two prominent addresses (8.8.8.8 and 8.8.4.4). By
06:00 the same morning, the DNS blocking had been removed across all
ISPs. Instead, to buttress the restrictions, providers shortly began to
drop all outgoing traffic to IP addresses associated with the twitter.com
domain, regardless of port or provider (\textbf{Event B}). By 16:00 of
that day, no Atlas probe could directly negotiate an SSL connection with
Twitter until the removal of the ban nearly two weeks later.

On March 27 (\textbf{Event C}), after recordings were posted of Turkish
national security officials discussing possible military action against
Syria, YouTube was blocked through false DNS answers for the youtube.com
domain. Within the Atlas network, this restriction appears as a slow
decline in the number of probes able to establish a connection to the
platform. However, unlike Twitter, a significant minority of probes
remained able to communicate with YouTube. Google's intertwined
infrastructure presents risk of collateral damage with network prefix
restrictions, which were not present with Twitter. Thus, clients that
were able to receive a valid address could reliably bypass the ban.

Beginning March 28, Turkish probes began to fail to establish SSL
connections to torproject.org (\textbf{Event D}). However, this
restriction neither included IP blocking, nor apparent interference with
the accessibility of the actual Tor network. Atlas probes could continue
to negotiate valid connections to Tor's directory authories. Throughout
the increased manipulation of local DNS services, nearly half of the
Atlas probes remained connected due to their use of foreign DNS services.

Later in the evening, March 28, hosts querying foreign-based DNS servers
began to receive the same false answers as those provided domestically,
leading to a rapid drop in availability of YouTube and Tor
(\textbf{Event E}). A publicly-available traceroute scheduled by
third-parties on the Atlas network against Google Public DNS returned
idiosyncratic and spontaneous shifts in Turkey's network topology timed
with these changes. This appears within traceroutes as a shortening in
the number of hops to Google, with a multifold reduction in traffic
latency and the absence of international hosts in path. The core
telecommunications provider T\"{u}rk Telekom had begun to reroute
traffic destined for Google to a local DNS server serving false answers.
Only TEKNOTEL Telekom maintained consistently valid routes for Google,
through Telecom Italia Sparkle. However, two days later Doruk
\.{I}leti\c{s}im and Net Elektronik Tasar{\i}m reestablished
connectivity through Euroweb Romania, circumventing upstream
interference. T\"{u}rk Telekom's redirection was finally removed late on
April 7.

By April 3, despite continued hijacking of Google Public DNS and
interference with YouTube, Twitter was unblocked for all probes
(\textbf{Event F}).  The total measurement credits we spent in order to conduct
this experiment are shown in Table~\ref{table:tr-costs}.

\begin{table}
    \begin{tabular}{l c c c c}
        \textbf{Target} & \textbf{Type}  & \textbf{Probes}  & \textbf{Freq (s)}  & \textbf{Credits}\\
        \hline
		 Twitter & SSL & 10 & 3,600 & 2,400 \\
		 YouTube & SSL & 10 & 3,600 & 2,400 \\
		 Tor & SSL & 10 & 3,600 & 2,400 \\
		 Twitter & DNS (U) & 10 & 3,600 & 2,400 \\
		 YouTube & DNS (U) & 10 & 3,600 & 2,400 \\
		 Twitter & Tracert & 10 & 3,600 & 7,200 \\
        \hline
        \multicolumn{4}{l}{\textit{Total (Daily)}}  & 19,200\\
        \multicolumn{4}{l}{\textit{Probes required}} & 0.89\\
        \hline        
    \end{tabular}
    \caption{Cost of measurements for Section~\ref{sec:turkey}.}
    \label{table:tr-costs}
\end{table}

% % % % % % % % % % % % % % % % % % % % % % % % % % % % % % % % % % % % % % % %
\subsection{Private sector cooperation in Russian filtering of Alexei Navalny}
\label{sec:russia}
% % % % % % % % % % % % % % % % % % % % % % % % % % % % % % % % % % % % % % % %

On March 13, 2014, Russia's Federal Service for Supervision in the
Sphere of Telecom, Information Technologies and Mass Communications
(Roskomnadzor) ordered the blacklisting of opposition figure Alexei
Navalny's LiveJournal blog.

At the same time, independent media portals were filtered, including the
news site grani.ru~\cite{ibtimes2014russia}.  Similar to Turkey,
Internet filtering in Russia is frequently conducted by IP blocking and
DNS poisoning~\cite{rugovdns,verkamp2012inferring}.  However, with a
random sample of 255 probes across 147 ASNs in Russia, only 38 probes on
20 ASNs received \emph{aberrant} DNS answers for Grani. Within this subset,
probes received a diverse, consistent selection of \emph{ten unique
addresses}, including two within private network address space
(10.52.34.222 and 192.168.103.162). A greater selection, 40 probes
across 23 ASNs, of traceroutes to port 80 for the primary address
associated with Grani (as of April 30, 23.253.120.92) failed within
Russia network space. 

In contrast to Grani, a locally-resolved DNS query for
navalny.livejournal.com over 255 probes on 146 ASNs received a
consistent reply of 208.93.0.190, which matched answers internationally
with only one anomalous response, a formerly valid address. The blocking
of Navalny's blog must be different from Grani. While the returned DNS A
record of 208.93.0.190 falls within a network prefix owned by
LiveJournal Inc. (208.93.0.0/22), over the 1,462 LiveJournal subdomains
in Alexa's Top 1 million list, 1,450 blogs resolved to another address,
208.93.0.150. Based on requests made independently of the Atlas network
from Europe, both hosts appear to be front servers for the LiveJournal
platform, as they return the same SSL Certificate and content. Requests
to 208.93.0.150 with a HTTP Host header set to navalny.livejournal.com
retrieves the correct content, and non-blacklisted content is
retrievable through 208.93.0.190.

As of April 2014, only five subdomains on livejournal.com could be found
whose DNS A records resolved to the address 208.93.0.190, Table
\ref{table:lj-blocked-blogs}, four of which are listed within Alexa's
top sites. All the blogs found on this alternative host have been
publicly declared by Russian authorities as in violation the country's
media laws for the promotion of political activities or extremism, and
two are listed within publicly-available filter site lists. 

\begin{figure*}
  \includegraphics[width=\textwidth]{diagrams/atlas_cache-results-measurement_id-1663748.png}
  \caption{Rostelecom's (AS12389) hijack of grani.ru Traffic in April 2014.}
  \label{image:ru-grani-hijack}
\end{figure*}

Based on timing, filtering lists, available domain names records, and
Atlas network measurements, it appears that a host was specially
established to faciliate Russian restrictions on content within the
LiveJournal platform. Using HTTPS Ecosystem Scans as a metric of
accessibility \cite{projectsonar}, the LiveJournal frontend at
208.93.0.190 came online between February 10 and February 17, with the
address otherwise unused until then. Two months later, the Ukrainian
LiveJournal blog `Pauluskp' (pauluskp.livejournal.com), which had
covered Russian involvement in Crimea, was filtered with the
administrative order listing an IP Address of 208.93.0.190. However, as
recently as six days before, the blog was recorded as pointing to the
main LiveJournal host. Similarly, the movement of Navalny's blog was
noticed within social media \cite{miptru2014}. It appears that in the
lead up to or at the time of filtering orders, LiveJournal coordinates
with authorities to alter the DNS A record for blogs designated by
Roskomnadzor, in order to segregate blacklisted content from the rest of
the platform.

\begin{table}
    \begin{tabular}{l c c}
        \textbf{Subdomain} & \textbf{Language} & \textbf{Roskomnadzor}\\
        \hline
        drugoi-nnover & Russian & Yes\\
        m-athanasios & Russian & Yes\\
        imperialcommiss & Russian & Yes\\
        pauluskp & Russian & Yes \\
        navalny & Russian & Yes \\
        \hline
    \end{tabular}
    \caption{LiveJournal DNS A Records of 208.93.0.190.}
    \label{table:lj-blocked-blogs}
\end{table}

Segregated LiveJournal content and blacklisted addresses are subject to
an additional, unknown method of network-layer interception performed
within the backbone network of Rostelecom (AS12389). While blog content
is not accessible over HTTPS, frontend hosts for LiveJournal offer SSL
services for the purpose of securing the transmission of user
credentials. On April 28, 78 of 343 Russian probes returned either
irregular responses or failed to connect to the alternative LiveJournal
host by address. Of this subset, 40 probes on 29 ASNs returned SSL
certificates with common name or locations fields attributed to Russian
ISPs. Based on HTTPS data, the four aberrant certificates captured have
been seen previously on seven Russian addresses belonging to the State
Institute of Information Technologies, Rostelecom and Electron Telecom
Network. Three of these hosts are responsive by their alternative,
public address and still match certificates. Two are generic ISP
homepages and one notifies of the blocking of the site `rutracker.ru.'
Other measurements that are unresponsive could be indicative of port
blocking or the redirection of traffic to a server that is not listening
for SSL connections.

The invalid certificates indicate that an intermediary in transit has
redirected the traffic out of its expected path to a third-party server
controlled by Russian entities. This approach is different from the
normal man-in-the-middle injection of responses seen in countries such
as Iran and Syria, and highlights the potential for Russian ISPs to
falsify content or gather user credentials. The observed behavior is not
limited to protocol or port, although the end host appears to be only
responsive to TCP requests, Figure~\ref{image:ru-grani-hijack}. This
holistic interference across Rostelecom's downstream peers suggests
redirection at the network layer, rather than application-based
classification of traffic associated with deep packet inspection.
Moreover, adjacent addresses within the same network, such as the normal
frontend for LiveJournal, traverse a valid international path. Instead,
blacklisted traffic appears to be coerced into a path controlled by
Rostelecom, indicating a narrowly-crafted interference with normal
routing through false advertisements or forwarding.
