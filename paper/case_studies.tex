\section{Case studies}

% here we need to add the turkey example. Collin you can filled this part. Philipp I remember you talked about the Tor stuff. Please don't do anything new.
% instead spend time collect and analyse what we have.

Consensus Building
--------------------



Timeline of Events
--------------------

Beginning March 20, 2014, social media users in Turkey began to report the selective availability of Twitter across the country's Internet Service Providers. YouTube and Twitter had both become increasingly the target of condemnation by Prime Minister Erdogan stemming from their use to distribute audio recordings and documents alleging financial and political impropriety by governmental officials ahead of local elections. Based on outstanding court complaints ordering the removal of content and users deemed defamatory, the national Information and Communication Technologies Authority (BTK) mandated the filtering of Twitter. Turkey's Internet filtering has previously been established to be based on DNS tampering and IP blocking \cite{akdeniz2010report}, which falls under the measurements possible through Atlas. Given both the governmental rhetoric preceding the ban, and the public's reaction after, period measurements were queued for the DNS, SSL, and Traceroute reachability of Twitter, YouTube, Google Public DNS and the Tor Project, which outline eight shifts in the BTK's content restrictions within a two week period.

While the BTK has been found to have used both local DNS manipulation and IP blocking for content restriction, it appears that the former is more commonly used by authorities. As of April 24, 2014, the Turkish-language filter monitoring site \textit{Engelliweb}, which tracks blocked content, only lists 167 IP addresses restricted in country, compared to 40566 domain names. The popularity of content distribution networks and shared hosting makes it possible for address-based blocking to easily be either overbroad, creating collateral damage from restricting all content on a host, or inefficiently narrow. Iranian authorities faced an aspect of this difficulty in 2012 when attempts to block addresses associated with YouTube was blamed for week-long disruptions on access to Gmail, a result of shared infrastructure across Google's services \cite{bbc2012gmail}. Foreign DNS servers quickly became both a circumvention mechanism, as well as a political statement, with the addresses of alternative services offered by Google and OpenDNS reportedly graffitied across the the country.

The ease and rate at which users were bypassing the filters, by some indications more tweets were sent after the restrictions than after, prompted broader crackdown by authorities. In the morning of March 22nd (\textbf{Event X}), between 01:00 and 02:00, ISPs began blocking Google's Public DNS service based on IP address, which would break normal browsing to filtered and non-filtered sites alike through interfering with the translation of names to Internet-routable addresses. Using Google's DNS is fairly normal, and not necessarily indicative of intent to circumvent filters; by some metrics 18.3\% of Internet users in Turkey rely on Google Public DNS \cite{ispcolumn2013googledns}. Within hours, by 6:00 the same day, the DNS blocking had been removed across all ISPs. However, to reinforce the restrictions, shortly after providers began to drop all outgoing traffic to addresses associated with the twitter.com domain regardless of port or protocol (beginning at 13:00), made easier based on the small number of addresses and minimal request of incurring overblocking. By 16:00 of that day, no Atlas probe could negotiate an SSL connection with Twitter.

On March 27 (\textbf{Event X}), after audo recording were posted of Turkish national security officials discussing military action against Syria, YouTube was blocked through DNS means. However, Google's infrastructure presents substantial risk of collateral damage on IP address or network prefix restrictions, and thus clients that were able to receive a valid address could bypass the ban.

On the evening of March 28th, hosts using foreign-based DNS servers began to receive the same false answers as those provided by local ISPs, leading to a further decrease in reported availability of YouTube. A publicly-available measurement scheduled on the Atlas network for traceroute measurements against Google's Public DNS over ICMP on a half-hourly basis from every probe returned idiosyncratic and spontentious shifts in Turkey's network topology. The implementation of the March 22nd address blockage of Google Public DNS falsely appears based on timing metrics to be a route hijack. While retaining valid routes across the international frontier into Google's network, the round trip times for several probes dropped precipitiously preceeding their blockage. For one of the last probes blocked, the time taken to reach the last hop out of the country had declined from 60.893 ms to 23.164 ms. However, by March 28, it became clear that telecommunications provider Turk Telekom had begun to reroute traffic destined for Google (\textbf{Event X}) to a local DNS server. Given that there does not appear to be a Google Public DNS front in country, these changes appeared as a shortening in the number of hops to Google, a reduction in traffic latency and the absence of international hosts in path.

Although the BTK was previously rumored to have ordered the installation of deep packet inspection equipment for purposes of lawful interception, subsequent restrictions against circumvention and anonymizations tools appears to have been limited to the filtering of websites \cite{kirlidog2011deep}. Beginning March 28th 19:00, Turkish probes began fail in attempts to establish an SSL connection to torproject.org, however, during this time hosts could continue to negotiate valid connections to the Tor network's Directory Authories.

By April 3th, despite continued hijacking of Google's Public DNS for filtering purposes and interference with YouTube, Twitter was unblocked for all probes.

Lessons from Blocking Event
--------------------

The BTK and Internet Service Provider's attempts to restrict access to highly popular social media platforms demonstrated the fragility of the country's filtering apparatus, as well as the perspective of Atlas within the network. Turkey's censorship is administratively and technically decentralized, and delays or irregulatory in the adoption of practices of particular strategies sheds light on the inconsistences of the application of BTK and other administrative orders.

Based on Atlas's view, several networks administered under Turk Telekomunikasyon had begun to filter YouTube by the time that measurements had been queued at March 21, with additional components networks of TTNet indicating attempts to censor within the following days. The differences of implementation of filtering rules is in part a product of Turkey's decentralized network infrastructure, differing from states such as Iran and Syria, which maintain public control over a limited number of international gateways. Therefore, content restrictions appear to be instituted by court or administrative order, which they comply at differing rates. Therefore, while YouTube's filtering was note widely noticed until the BGP hijack events, the lag between quickly compliant service providers may provide opportunities for early warning for intent to filter.

Additionally, the filtering apparatus appears to not to have been designed to handle widespread intent to circumvent, particularly when more aggressive restrictions could incur collateral damage. Upon initial DNS restrictions, only 20\% of probes in Turkey were no longer able to connect to Twitter. The remainder either utilized foreign DNS servers or tunneled traffic out of country by unknown means, thereby creating inconsistent reports of accessibility. This may reflect a common experience for a large portion of the Turkish population, given previous accounts of the adoption of foreign DNS servers. Direct access to Twitter was only effectively cut off by March 23, when traffic to the platform was firewalled by providers. However, even with the BGP hijack and local DNS interference, YouTube remained accessible for 40\% of Atlas probes attributed to Turkish networks. Turkey's filtering apparatus is more effective at mitigating ubiquitious access to marginal content or coercing content providers into complience, than actual denial of access to information for a sufficiently motivated user.

From purposes of measurement, due to the Turkish telecommunications companies reliance on network reachability, rather than traffic inspection, the blocking incident was well positioned for research through Atlas probes. However, without secondary investigation, Atlas initially produced results that would call into question the veracity of accounts of Twitter's filtering. This reflects a disparity between the network conditions of the probes and those of the average user. At the time of blocking, 70\% (DOUBLE CHECK) of probes relied on DNS server that was outside of the country, and at least one had at various times tunneled traffic outside of the country. Moreover, if authorities had utilized HTTP inspection, Atlas would not have been capable of documenting the event.

Finally, potential collateral damage and level of difficult appears to have shaped the implementation of Turkey's filtering mandates. The quick removal of restrictions Google Public DNS, and then later attempts to impersonate the service, was indicative that absolute prohibition was not worth incurring the cost of breaking access for a significant portion of the population. While alternative strategies were possible with Twitter, due to its addressing schema, historical lessons from other countries' attempt to filter YouTube has run into the complexity and interdependency of Google's services. It appears that rather than potentially cutting off access to all Google products, Turkish network administrators have allowed circumvention strategies to continue.

(Outstanding Tests)
*** Can I try to negotiate with all dir authorities?


Attempts to query sites known to be blocked in Iran receive correct DNS replies
\citation{halderman iran}
ssl connection fails however




Navalny Hijack
----------------------

On March 13, 2014, Russian's Federal Service for Supervision in the Sphere of Telecom, Information Technologies and Mass Communications (Roskomnadzor) announced the blacklisting of opposition figure Alexei Navalny's LiveJournal blog, due to his violation of the terms of his house arrest. Shortly after the order, Russian social media and state news agencies reported that access to the enitirity of LiveJournal had been temporarily restricted on some ISPs due to their inability to distinguish traffic to different blogs [1, 2]. When the regional ISP Netis Telekom had previously blocked access to all LiveJournal blogs, while attempting to comply with a court order against a neo-nazi blog, the platform's Russian Director Ilya Dronov attributed the overblocking to the ISP's reliance on IP-based filtering and the inclusion of IP addresses within court orders[2]. Similar to Turkey's filtering apparatus, Russia's content restrictions are primarily reliant on DNS poisoning and IP filtering [3, 4], presenting the same potential weaknesses and circumvention strategies.

[1] http://www.russiaslam.com/2014/stories/censors-block-access-to-livejournal-because-of-navalnys-blog.html
[2] http://en.itar-tass.com/russia/723571
[3] http://www.leta.ru/press-center/news-ib/news-ib_894.html


Concurrently, four opposition news portals were blocked due to based on the allegations that they called for ``illegal activity and participation in mass events that are conducted contrary to the established order,'' including grani.ru. Only two probes known to domestic domain name services received valid DNS replies, the remainder XX spanning XX ASNs received false answers that included ten unique addresses, including two addressesin with RFC1918, private network address space (10.52.34.222 and 192.168.103.162). In contrast to grani.ru, navalny.livejournal.com recieved the consistent and valid reply of 208.93.0.190 across all providers. However, while the returned A record of 208.93.0.190 falls within an address prefix 208.93.0.0/22 owned by Live Journal Inc., this responses differs from the type response for hosted blogs of 208.93.0.150. Both addresses appear to front servers for the LiveJournal platform as they return the same SSL Certificate and host the same content. Requests to 208.93.0.150 with a HTTP Host header set to navalny.livejournal.com retrieves the correct content and non-blacklisted content is retrievable through

Based on timing, publicly-available history, available domain names records, and Atlas network measurements, it appears that a host was specially established to faciliate Russian censorship of the LiveJournal platform, potentially based on interest by the service to mitigate further overblocking of the platform. As of April 2013, only five subdomains of livejournal.com could be found whose DNS A records resolved to the address 208.93.0.190 (hence labeled as Alternative-Front -- OR SOMETHING), Figure \cite{lj-blocked-blogs}. All the blogs found have been publicly cited as having been found to violate Russia's media laws for promotion of political activities or extremism, and two are listed within publicly available filter site lists [7]. Within the 1462 Live Journal addresses with Alexa's Top 1 million list, 1450 blogs reolved to the 208.93.0.150 (Main-Front), another four of those addresses being amongst the five previously identified. Using HTTPS Ecosystem Scans as a metric of accessibility[5], the frontend available on 208.93.0.190 came online between February 10 and February 17, with the address otherwise unused until then. On April 11, 2014 the LiveJournal site of Ukrainian blogger `Pauluskp' (pauluskp.livejournal.com), which had covered Russian involvement in Crimea, was filtered with the administrative order listing an IP Address of 208.93.0.190, however, as recently as April 5th, the blog was recorded as pointing to the Main-Front. Similarly, the movement of Navalny's blog was noticed within social media [6] In the lead to filtering orders, it appears that LiveJournal coordinates with authorities and alters the DNS A record for blogs designated by Roskomnadzor in order to faciliate ISP implementation of government restrictions [6].

While blog content is not accessible over HTTPS, LiveJournal hosts appear to offer SSL services for presumably e a limited purpose of securing user credentials. As of April 28, only 12 of 90 probes within Russia were able to establish a successful SSL connection to Navalny's blog, a further 38 returned errors potentially indicative of port blocking. The plurality, 40 probes on 29 ASNs, returned SSL certificates with common name or locations fields attributed to Russian ISPs, indicating that an entity in transit had intercepted or redirected the traffic out of its normal path. This approach is different from the normal man-in-the-middle approach seen in countries such as Iran and Syria. Additionally it is not limited to protocol, although the end host appears to be only responsive to TCP requests, and is narrowly tailored to only the address associated with the Alternative-Front. Adjacent addresses and others destinations, such as the normal frontend, within the same prefix appear to traverse a valid international path.


\begin{table}[h!]
    \label{lj-blocked-blogs}
    \begin{tabular}{| c |}
        URL\\
        drugoi-nnover.livejournal.com \\
        imperialcommiss.livejournal.com \\
        m-athanasios.livejournal.com \\
        navalny.livejournal.com \\
        pauluskp.livejournal.com \\
    \end{tabular}
\end{table}


[3] http://rkn.gov.ru/docs/Analysys_and_recommendations_comments_fin.pdf
[4] https://www.usenix.org/sites/default/files/conference/protected-files/verkamp_foci12_slides.pdf
[5] Project Sonar: IPv4 SSL Certificates
[6] http://board.rt.mipt.ru/?read=8820778



