\section{Discussion}

\subsection{Ethical Considerations}

% TODO Roya will fill this up later.
The problem: we might endanger probe contributors by running censored queries over their probes.
Contributors probably don't expect that their probe is used for censorship analysis.

Our justification: Plausible deniability, we are limited to X.509/DNS/ping/traceroute.  We are not
claiming that everything is OK.  Instead, we encourage further discussion.

\subsection{Conclusion}

% The implementation of the March 22nd address blockage of Google Public DNS falsely appears based on timing metrics to be a route hijack. While retaining valid routes across the international frontier into Google's network, the round trip times for several probes dropped precipitiously preceeding their blockage. For one of the last probes blocked, the time taken to reach the last hop out of the country had declined from 60.893 ms to 23.164 ms. However, by March 28,

These findings contribute to broader discussions on anticensorship strategies.  Collateral damage and level of difficulty appears to have shaped the implementation of Turkey and Russia's filtering mandates and responses to circumvention. The quick removal of restrictions on Google Public DNS, and then later attempts to impersonate the service, demonstrated that enforcing an absolute prohibition on filtered content was not worth incurring the cost of disrupting access a large portion of the population. While alternative strategies were possible with Twitter, due to its addressing schema, historical lessons from other countries' attempt to filter YouTube has run into the complexity and interdependency of Google's services. Where there are high collateral costs, such as with YouTube in Turkey and LiveJournal in Russia, authorities have been forced to either limit their restrictions or find cooperative arrangements with the platform. 

Filtering apparatuses may be more effective at disrupting easy access to marginal content or coercing content providers into compliance, than actual denial of information for a sufficiently motivated user. Neither Turkey or Russia's filtering apparatuses appears to have been designed to handle widespread intent to circumvent, particularly when more aggressive restrictions could incur collateral damage. When foreign DNS requests are allowed and not subject to deep packet inspection, circumvention is simple. Upon initial DNS restrictions, only 20\% of probes in Turkey were no longer able to connect to Twitter. The remainder either utilized foreign DNS servers or tunneled traffic out of country by other means, thereby creating inconsistent reports of accessibility. This may reflect a common experience for a large portion of the Turkish population, given previous accounts of the adoption of foreign DNS servers. Direct access to  was only effectively cut off by March 23, when traffic to the platform blocked by IP address. Even with the BGP hijack and local DNS interference, YouTube remained accessible for 40\% of Atlas probes attributed to Turkish networks. 

Finally, due to Turkish and Russian telecommunications companies reliance on blocking network reachability and interfereing with name translation, rather than traffic inspection, the Atlas network was well positioned for both blocking incident. If authorities had utilized HTTP inspection, Atlas would not have been capable of documenting either event. However, without secondary investigation and additional restrictions, initially produced results would have called into question the veracity of accounts of Twitter's filtering. Other inconsistences remain unaccounted for. 

In Atlas's view, several networks within Turk Telekomunikasyon had already began to filter YouTube by the time measurements had been queued at March 21. These measurements precede public accounts and broader censorships by the several days. The differences of experiences of filtering rules is in part a product of Turkey's decentralized network infrastructure, differing Iran and Syria, which maintain monopolistic over international gateways. Content restrictions appear to be instituted by court or administrative order, which are complied with at differing rates. However, this also reflects a disparity between the network conditions of the probes and those of the average user. At the time of blocking, 70\% of probes relied on a DNS server outside of the country, and at least one had at various times tunneled all traffic by unknown means. Moreover, while less than half of measurements within Russia show signs of interference, censorship of the content and manner tested appears to be a persistant experience. Atlas does not reflect the experience of the public portionally based on numerical results, the 20\% of probes experiencing traffic misdirection by Rostelecom is likely representative of a substantially larger percentage of users.

Despite these analytical precautions, RIPE Atlas based measurement provides an early perspective in the opportunities and methodologies possible with pervasive censorship research. Previous examinations of the methods and themes of Internet filtering have tended to analyze specific apparatuses on a per-country basis, assuming internal consistency. This approach has been appropriate for describing the diversity of methods used to control access globally, as well as for when the primary focus is on content themes and in countries that impose restrictions at central points of transit. However, as Internet censorship has increased in locations with heterogentity and private markets at the international frontier, neither compliance strategies nor accountability can assume direct and homogenous control by authorities. Russia and Turkey's networks are more administratively and technically decentralized than China and Iran. The delays or irregulatories in the adoption of practices of particular strategies sheds light on the inconsistences of the application of administrative orders that are applicable to censorship measurement. Our initial research demonstrates that across national networks there are substantive differences of methods, rates of implementation, and, in at least one case, even selective compliance of mandates for content restrictions that are measurable by future initiatives.
