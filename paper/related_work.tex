\section{Related Work}
% This section will review three topics: first current existing direct network measurment tools such as
% Dimes, planet lab ,... Second Talks about other ways to do censorship analysis which is pretty much
% idle scanning and using side channels ... Third: we also need to cite recent papers/articles using
% Ripe for Internet measurment studies...
% TO DO: Philipp and Roya ...

% Longitudinal studies.
It is not difficult to conduct one-off censorship studies because censors typically do not have
sufficient time to react and thwart the study.  Longitudinal studies, on the other hand, are more
challenging as they have to be designed in a tamper-proof way.  In 2007, Crandall et al.  proposed
ConceptDoppler~\cite{Crandall2007}.  The design enables longitudinal censorship analysis by
detecting which keywords are filtered by the Great Firewall of China (GFW) over time.  More
recently, CensMon was introduced by Sfakianakis et al. in 2011~\cite{Sfakianakis2011}.  CensMon is a
web censorship monitor which is run on top of PlanetLab~\cite{planetlab}.  Most recently,
Filast\`{o} and Appelbaum presented OONI~\cite{Filasto2012} in 2012.  In contrast to CensMon and
ConceptDoppler, OONI is deployed and has been used successfully.\footnote{Gathered reports are
available online \url{https://ooni.torproject.org/reports/}.}  In parallel to all open platforms
discussed above, proprietary platforms exist~\cite{herdict,opennet} but we will not discuss them as
their closed nature does not encourage analysis.
