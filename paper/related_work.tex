\section{Related work}
\label{related_work}
\begin{table*}[ht!]
\centering
\begin{tabular}{l|cccc}
\textbf{Platform} & \textbf{Flexibility} & \textbf{Coverage} &
\textbf{Blocking resistance} & \textbf{Main use} \\
\hline 
PlanetLab~\cite{planetlab} & High & Low/Medium & Medium & Network measurements \\
Atlas~\cite{atlas} & Low & Medium/High & Medium & Network measurements \\
M-Lab~\cite{dovrolis2010measurement} & Low & High & Medium & Network measurements \\
Tor~\cite{Dingledine2004} & Medium & Medium & Low & Low-latency anonymity \\
OONI~\cite{Filasto2012} & High & Low & Medium & Interference analysis \\
Herdict~\cite{Herdict} & Low & Low/Medium & Low & Interference analysis \\
OpenNet~\cite{opennet} & Low & Medium & High & Interference analysis \\
\hline 
\end{tabular} 
\caption{Comparison between several popular filtering analysis platforms.}
\label{tab:comparison}
\end{table*}

% Longitudinal studies.
It is not difficult to conduct one-off studies on filtering because administrators and governments
typically do not have sufficient time to react and thwart the research.
Longitudinal studies, on the other hand, are more challenging as they have to
be designed in a tamper-proof and sustainable way.  In 2007, Crandall et al.
proposed ConceptDoppler~\cite{Crandall2007}.  The design enables longitudinal 
analysis by detecting which keywords are filtered by the Great
Firewall of China (GFW) over time.  More recently, CensMon was introduced by
Sfakianakis et al. in 2011~\cite{Sfakianakis2011}.  CensMon is a web censorship
monitor which is run on top of PlanetLab~\cite{planetlab}.  In
2012, Filast\`{o} and Appelbaum presented OONI~\cite{Filasto2012}.  In
contrast to CensMon and ConceptDoppler, OONI is deployed and has been used
successfully.\footnote{Gathered reports are available online at:\\
\url{https://ooni.torproject.org/reports/}.}  In parallel to these measurement tools 
are centrally-maintained platforms and proprietary collection agents~\cite{hwang2007herdict,opennet}.

% Comparison to other projects.
Table~\ref{tab:comparison} contains a comparison between popular and deployed
platforms that are or can be used for analysis of information controls.  Our comparison is
based on \emph{flexibility} (i.e., how many types of measurements can be run),
\emph{coverage} (i.e., how many probes in how many countries are available),
and \emph{blocking resistance} (i.e., how easy it is for network intermediaries to disable the
respective platform).  We qualitatively compare all platforms and assign them
the labels ``Low'', ``Medium'', or ``High''.  Note that we do not propose
Atlas as \emph{replacement} for any existing measurement platforms.
Instead, we see it as a \emph{complement} that contributes to the already
existing and growing landscape of initiatives.

% Side channel measurements.
Additionally, in the absence of deployed platforms or other means to access
machines inside countries of interest, analysts have resorted to
exploiting TCP/IP side channels.  In particular, Ensafi et al. demonstrated
how to measure intentional packet dropping without controlling
either the source or the destination machine~\cite{Ensafi2014}.

% Previous Atlas censorship analysis.
Atlas has already been used as platform for analysis of network
disruptions outside an academic setting.  In 2014, Maass used Atlas to
find inconsistencies in the DNS records and X.509 certificates for
torproject.org~\cite{Maass2014}.  In the same year, Bortzmeyer and Aben
independently discussed service interference in
Turkey~\cite{Bortzmeyer14,Aben14}.  While we discuss the same topic in
Section~\ref{sec:case_studies}, we do so with significantly more data
and in a more rigorous fashion.
