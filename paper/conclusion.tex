\section{Conclusion}
\label{sec:conclusion}

<<<<<<< HEAD
% TO DO: Do we want to use nymity.ch or cartography.io (which I assume would just be a Github page)?

In this paper, we have presented a model of an interference detection platform that builds on top of the RIPE Atlas community. Previous examinations of Internet filtering have tended to analyze specific national apparatuses on a per-country unit, assuming internal consistency across providers and time. This past approach has been appropriate for describing the diversity of methods used to control access globally, as well as for when the primary research focus is countries that impose restrictions at central points of international transit. 

As Internet filtering has proliferated to countries with competition and private markets at the international frontier, researchers can no longer assume direct and homogenous control by authorities. Our application this model against two recent and developing cases of interference in Russia and Turkey demonstrate this shifting environment. Russia and Turkey's networks are more administratively and technically decentralized than China and Iran~\cite{Roberts2011}. Subsequent, through longitudinal observation, our initial research demonstrates substantive differences of methods, rates of implementation, and in at least one case, even selective compliance for content restrictions. 

In both Turkey and Russia, the Atlas network provided a unique opportunity for documenting rapidly evolving information controls due to its nearly ubiquitous geographic presence, stability for recurrent measurements, and external queuing of targets. Reliance on alternative models outlined in Section~\ref{related_work} would have imposed delays on deployment of measurements and limited the number of vantage points from which data could be collected. This enabled partial identification of the diversity of blocking strategies employed, found cheating by ISPs affected by upstream interference and provided early warning where networks implemented administrative decisions before others.

These findings contribute to broader discussions on anti-filtering strategies.  Collateral damage, urgency and level of difficulty appears to have shaped the implementation of Turkey and Russia's filtering mandates and responses to circumvention. The quick removal of restrictions on Google Public DNS, and then later attempts to impersonate the service, indicate that enforcing an absolute prohibition is partially an economic question. While alternative strategies for filtering were possible with Twitter, due to its addressing schema, historical lessons from other countries' attempt to filter YouTube has run into the complexity and interdependency of Google's services. Where there are high collateral costs, such as with YouTube in Turkey and LiveJournal in Russia, authorities appear to have limited their restrictions or found cooperative arrangements with platform owners. 

The Atlas network was well positioned for documentation of both blocking incidents based on Turkish and Russian telecommunications companies reliance on interfering with network reachability and domain name translation. If administrators had utilized traffic inspection, or more subtly degraded connectivity without outright blocking access, Atlas would not have been capable of documenting either event.

Despite these analytical precautions, Atlas-based measurements provides an early perspective on the opportunities and methodologies concomitant with pervasive network observation. We document a heterogeneous filtering infrastructure in both countries, notably reliant on DNS manipulation and redirection of traffic by transit providers. The latter manipulation of network routes represents an underexplored form of interference and invokes the need for measurement tools to collect path information to complement other forms of documentation. Furthermore, the delays or irregularities sheds light on the inconsistences of the application of administrative orders that are applicable to censorship measurement. Our initial research demonstrates that across national networks there are substantive differences of methods, rates of implementation, and, in at least one case, even selective compliance of mandates for content restrictions that are measurable by future initiatives.

Our code and datasets are available online at \url{http://cartography.nymity.ch}.
=======
In this paper, we evaluated the RIPE Atlas platform for censorship analysis.
We

Our measurements and analysis provide an early perspective on the opportunities
and methodologies possible with pervasive censorship research. Previous
examinations of Internet filtering have tended to analyze specific apparatuses
on a per-country basis, assuming internal consistency. This past approach has
been appropriate for describing the diversity of methods used to control access
globally, as well as for when the primary research focus is on content themes
and in countries that impose restrictions at central points of transit. 

However, as Internet censorship has increased in countries and polities with
competition and private markets at the international frontier, researchers can
no longer assume direct and homogenous control by authorities. Our applications
of Cartography against two recent and large-scale censorship incidents in
Russia and Turkey demonstrate this. Russia and Turkey's networks are more
administratively and technically decentralized than China and
Iran~\cite{Roberts2011}. Delays or irregulatories of implementation sheds light
on the inconsistences of the administration of censorship. 

Our initial research demonstrates that across national networks there are
substantive differences of methods, rates of implementation, and, in at least
one case, even selective compliance for content restrictions. Moreover, the
available measurements of Atlas proved sufficient for Cartography to uncover
interesting and evolving censorship practices in jurisdictions that are only
beginning to assert expansive control of online content.

Our code is available online at \url{http://cartography.nymity.ch}.
>>>>>>> FETCH_HEAD
