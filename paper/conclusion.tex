\section{Conclusion}
\label{sec:conclusion}

In this paper, we have presented a model of an interference detection
platform that builds on top of the RIPE Atlas community. Previous
examinations of Internet filtering have tended to analyze specific
national apparatuses on a per-country unit, assuming internal
consistency across providers and time. This past approach has been
appropriate for describing the diversity of methods used to control
access globally, as well as for when the primary research focus is on
countries that impose restrictions at central points of international
transit. 

As Internet filtering has proliferated to countries with competition and
private markets at the international frontier, researchers can no longer
assume direct and consistent control by authorities. The two recent and
developing cases of interference in Russia and Turkey demonstrate this
shifting environment. Russia and Turkey's networks are more
administratively and technically decentralized than China and
Iran~\cite{Roberts2011}. Through longitudinal observation, our initial
research demonstrates substantive differences of methods ands rates of
implementation for content restrictions. In both, the Atlas network
provided a unique opportunity for documenting rapidly-evolving
information controls due to its nearly ubiquitous geographic presence,
stability for recurrent measurements, and external queuing of targets.
Reliance on alternative models outlined in Section~\ref{related_work}
would have imposed delays on deployment, and limited the vantage points
from which data could be collected. 

These findings contribute to broader discussions on anti-filtering
strategies.  Collateral damage, urgency and level of difficulty appears
to have shaped the implementation of Turkey and Russia's filtering
mandates. The quick removal of restrictions on Google Public DNS, and
then attempts to impersonate the service, indicate that enforcing an
absolute prohibition on content is partially an economic question. Where
there are high collateral costs, such as with Google infrastructure in
Turkey and LiveJournal in Russia, authorities appear to have limited
their restrictions or found cooperative arrangements with platform
owners. 

Atlas was well positioned for documentation of both blocking incidents
based on telecommunications companies reliance on interfering with
network reachability and domain name translation. If administrators had
utilized traffic inspection, or more subtly degraded connectivity
without outright blocking access, the platform would not have been
capable of measuring these events.

Despite these analytical precautions, Atlas-based measurements provide
an early perspective on the opportunities and methodologies possible
with pervasive network observation. We document multifaceted filtering
infrastructures in both countries, notably reliant on DNS manipulation
and redirection of traffic by transit providers. Additionally, the
latter manipulation of network routes represents an underexplored method
of interference and invokes the need for tools to collect path
information to complement other forms of documentation. Furthermore,
differences of restrictions shed light on  inconsistencies in the
application of administrative orders, and could provide early warning of
increased controls in the future. Our initial research demonstrates that
across national networks there are substantive differences of methods,
rates of implementation, and, in at least one case, even selective
compliance for information controls that are measurable by Atlas and
future initiatives.

Code and datasets are available online at: \url{http://cartography.io}.
