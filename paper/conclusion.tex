\section{Conclusion}
\label{sec:conclusion}

In this paper, we evaluated the RIPE Atlas platform for censorship analysis.
We

Our measurements and analysis provide an early perspective on the opportunities
and methodologies possible with pervasive censorship research. Previous
examinations of Internet filtering have tended to analyze specific apparatuses
on a per-country basis, assuming internal consistency. This past approach has
been appropriate for describing the diversity of methods used to control access
globally, as well as for when the primary research focus is on content themes
and in countries that impose restrictions at central points of transit. 

However, as Internet censorship has increased in countries and polities with
competition and private markets at the international frontier, researchers can
no longer assume direct and homogenous control by authorities. Our applications
of Cartography against two recent and large-scale censorship incidents in
Russia and Turkey demonstrate this. Russia and Turkey's networks are more
administratively and technically decentralized than China and
Iran~\cite{Roberts2011}. Delays or irregulatories of implementation sheds light
on the inconsistences of the administration of censorship. 

Our initial research demonstrates that across national networks there are
substantive differences of methods, rates of implementation, and, in at least
one case, even selective compliance for content restrictions. Moreover, the
available measurements of Atlas proved sufficient for Cartography to uncover
interesting and evolving censorship practices in jurisdictions that are only
beginning to assert expansive control of online content.

Our code is available online at \url{http://cartography.nymity.ch}.
