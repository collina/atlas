\section{Framework structure}
\label{sec:framework}
In order to assess Atlas's aptitude as an interference measurement platform, we
continue by presenting available data collection mechanisms and our analytical framework.

\subsection{RIPE Atlas background}
% Some basic information about Atlas.
Founded in 2010 by RIPE NCC, Atlas~\cite{atlas} is a globally
distributed Internet measurement network consisting of physical probes hosted
by volunteers.  Once a user connects her probe to the network, it can be used
by other participants for measurements. So-called \emph{credits} are awarded
automatically based on the uptime of contributed probes, which are expended in
order to perform custom measurements. Queries to probes can be initialized
centrally either over the web frontend, or over a RESTful API.

% Geographic and topological distribution.
An ideal measurement platform features high geographic and
topological diversity, thereby facilitating measurements in any region where
filtering occurs.  While Atlas probes are distributed throughout the world,
there is a significant bias towards the U.S. and Europe as can be seen in
Figure~\ref{fig:probe_distribution}.  As for Atlas's topography, only 68
autonomous systems contain 40\% of all Atlas probes with the three most common
autonomous system numbers being AS7922 (4.4\%, Comcast Cable Communications),
AS3320 (3.2\%, Deutsche Telekom), and AS6830 (2.8\%, Liberty Global Operations).
While not optimal, most regions of particular interest still contain at least several
probes.

\begin{figure}[t]
\centering
\includegraphics[width=0.48\textwidth]{diagrams/probe_distribution.jpg}
\caption{The geographic distribution of Atlas probes as of May 2014.  Green
icons represent active probes whereas red icons represent probes which are
currently offline.  The distribution is heavily biased towards the U.S. and
Europe.} \label{fig:probe_distribution}
\end{figure}

% What kind of measurements does Atlas allow?
As of May 2014, Atlas allows four types of measurements; ping, traceroute, DNS
resolution, and X.509 certificate fetching (henceforth called SSLCert).  All
four measurement types can further be parameterized for more fine-grained
control.  HTTP requests are not possible at this point due to abuse and
security concerns.  While Atlas clearly lacks the flexibility of comparable
platforms (see Table~\ref{tab:comparison}), it makes up for it with high
diversity, responsiveness, and continued growth.  After all, we do not expect
Atlas to replace existing platforms, such as OONI, but rather to
\emph{complement} them.

% We also need to mention how you can stablish a big experiments using RIPE.
% And mention about the propblem of how you frist need to submit your
% experiments and then it will give you predicted cost. This way we can move to
% cost function.

\subsection{Atlas's cost model}

As previously mentioned, Atlas measurements are paid with platform credits.
The exact ``price'' of a measurement depends on the measurement type, its
parameters, and the number of destinations.  The credit system works based on a
linear cost model.  Each user has a credit balance that can be increased
steadily by hosting Atlas probes\footnote{As of June 2014, 21,600 credits per
day of uptime.} or by receiving credits from other users.

Table~\ref{tab:cost} lists the currently available measurement types as well as
their associated costs.  While DNS and SSLCert measurements have a fixed cost,
ping and traceroutes vary depending on the amount and sizes of packets.  Also,
one-off measurements cost twice as much as repeated measurements.  When
scheduling a new experiment, the user first specifies the details (e.g.,
measurement type as well as measurement parameters).  Afterwards, Atlas's
web-based frontend calculates the measurement costs on the server side and
shows it to the user.  Finally, upon completion of the measurement, the
respective cost is subtracted from the user's credit balance.

Due do the non-deterministic nature of pings and traceroutes, and measurements
in general, we developed a command-line based tool to help users create new
measurements and estimate their costs.\footnote{The tool is available at
\url{http://cartography.io}.}  As input, the tool expects \emph{1)}
a country of interest, \emph{2)} the amount of credits, the user is willing
to ``pay'', and \emph{3)} a measurement type.  Our tool then determines the
amount of available probes (if any), the expected costs, and runs the
measurement if the cost is below the user's expected cost.

%\subsection{Calculating Costs}
%
%Predicted cost = $\sum_{i=1}^{5} (C_i * N_i)$  where i can be (DNS, SSL, ...)\\
%Remaining credits = min(available cost, desired cost) - predicted cost
%
%Note that it is a linear cost model so number of probs included doesn't matter
%in the cost part it matters when we suggest the experiment by just uniformly
%distribute the possible number of measurements on one(total experiments
%possible/Num probs)
%This should be included in the API philipp wrote (TODO)
%
%one-off measurements cost twice as much as not one-off.

% \subsection{Evaluation of our cost model}
% To test that our cost prediction is accurate, we ran a series of unit cost with
% different values, and compared  the suggested cost and the actual credit
% subtracted with the cost our approach suggest. Because the number of possible
% cases are finite we can run all unit measurements, then calculate confidence
% interval to show how good we do.
% 
% We choose a prob(12214) 

\begin{table}[t]
\centering
\begin{tabular}{lc}
\textbf{Measurement} & \textbf{Cost in credits} \\
\hline 
DNS\slash DNS6 (TCP) & 20\\ 
DNS\slash DNS6 (UDP) & 10\\ 
SSLCert\slash SSLCert6 & 10 \\
Ping\slash Ping6 & $N * (int(S/1500)+1)$\\
Traceroute\slash Traceroute6 & $ 10*N*(int(S/1500)+1)$\\
\hline 
\end{tabular} 
\caption{The cost for all available Atlas measurements.  The variable $N$
refers to the number of packets whereas the variable $S$ refers to packet
sizes.}
\label{tab:cost} 
\end{table}

\subsection{Assessing measurement integrity}

Despite their distribution across a diversity of countries and networks,
RIPE Atlas may not fully reflect the Internet as it is experienced by
the general public, as probes neither fully emulate the network position
nor the configuration of an average user. As an immediate control,
efforts are taken to verify the reachability of non-controversial
content and identify whether probes use domestic domain name servers.
These probes may be excluded from measurements in order to avoid Type 1
and Type 2 errors.

Even with additional precautions, idiosyncratic observations are an
inevitable product of the high rate of placement of probes on commercial
and academic networks. These institutions may have alternative
connectivity that is faster and less highly regulated than consumer
networks. Additionally, disrupting or degrading connections based on
plain text data, traffic classification or application headers would fall
outside of the measurements currently possible with Atlas probes.
Lastly, Atlas, as with most measurements outlined within
Section~\ref{related_work}, is unlikely to detect content restrictions
imposed by the platforms themselves, such as search manipulation or 
withholding of content based on a user's location.

\subsection{Rough consensus validity}

The international distribution of web services, such as content delivery
networks, has created additional complexity in the determining whether
measurement results are genuine. While SSL and DNSSEC utilize
third-party trust to validate answers, Certificate Authorities have been
previously compromised by state and non-state actors, and DNSSEC is not
widely implemented. In order to validate answers within the Atlas
network, we use a cross-country comparison of results to queries. This
methodology assumes that intermediaries who interfere with connectivity
do not coordinate strategies internationally. States and service
providers impose different filtering approaches for purposes of
localization, infrastructure or even the monetization of blocked
traffic. Furthermore, interference is more effective when the public is
unaware of the practices and technologies employed against them, placing
a strong incentive on secrecy. Therefore, as a simple test of validity,
we count the number of countries or ASNs that an answer, such as a DNS A
record, final network of transit or a certificate hash, is seen. Any
response with fewer than the mean number of jurisdictions, or those
within private network spaces (RFC 1918~\cite{rfc1918}), are treated as
potentially aberrant and flagged for further investigation.

\subsection{Ethical aspects}

Atlas was not designed explicitly for analysis of information controls
and accordingly, its volunteers likely may not expect that their probes
will be used for such purposes.  Careless measurements could attract
attention and cause repercussions for probe operators.  In addition, an
increased used of Atlas for politically-sensitive analysis could scare
away probe operators and jeopardize the usefulness of the platform.
These concerns extend beyond merely complying with Atlas's acceptable
usage policies and guide our selection of measurements.

Atlas's measurement types are limited in scope.  As of May 2014, it is
not possible to create HTTP requests or engage in actual, meaningful
communication with arbitrary destinations, which limits the damage
caused by reckless measurement. We are not aware of environments where
low-level network queries to commonly frequented platforms or services
solicit attention from authorities, even when blocked, nor where answers
are falsified in order to stifle research. Requests for sites such as
Facebook are generated as a part of normal web use due to script
inclusion, and Google Public DNS is commonly used due to its
reliability. Commonplace sites are a different class of potential
monitoring targets than content promoting child abuse or violent
extremism.

The balance between research interests and exposure to risk is an area
of concern shared across all initiatives identified in
Section~\ref{related_work}. This has stimulated a broader discussion
that will play a factor in future utilization of Atlas. Nevertheless, we
stress that great care must be taken when planning measurements because
the volume and types of measurements could still be suspicious. We
believe that Atlas has a place in the domain of censorship analysis but
it has to remain a small place, lest it endangers users or the platform
itself. For a more comprehensive ethical discussion, see Wright et
al.~\cite[\S~5]{Wright2011}.
